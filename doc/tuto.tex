\documentclass{article}

%%%%%%%%%%%%%%%%%%%%%%%%%%%%%%%%%%%%%%%%%%%%%%%%%%%%%%%%%%%%%%
%%%% For Alectryon

\usepackage{texments}
%%% for movies by alectryon
\usepackage{./movies/snippets/assets/alectryon}
\usepackage{./movies/snippets/assets/pygments}

% Prevent breaks in the middle of syntactic units
\let\OldPY\PY
\def\PY#1#2{\OldPY{#1}{\mbox{#2}}}


%%% One hypothesis per line 
\makeatletter
\renewcommand{\alectryon@hyps@sep}{\alectryon@nl}
\makeatother

%%% \snippets{A,B,C,…} inputs a series of snippets as one block (with \itemsep
%%% between them).  A, B, C should be paths to files in snippets/.
\usepackage{etoolbox}
\makeatletter

\newcommand{\pathtomovies}{./movies}

\newcommand{\inputsnippets}[1]
  {{\setlength{\itemsep}{1pt}\setlength{\parsep}{0pt}% Adjust spacing
    \alectryon@copymacros\begin{io}
      \forcsvlist{\item\@inputsnippet}{#1}
    \end{io}}}
\let\input@old\input % Save definition of \input
\newcommand{\@inputsnippet}[1]
  {{\renewenvironment{alectryon}{}{}% Skip \begin{alectryon} included in snippet
    \input@old{{\pathtomovies}/snippets/#1}}}
\makeatother

% End of Alectryon stuff
%%%%%%%%%%%%%%%%%%%%%%%%%%%%%%%%%%%%%%%%%%%%%%%%%%%%%%%%%%%


\title{Lexicographic exponentiation: An example of proof documentation with Alectryon}

\author{Labri Coq working group}

\begin{document}
\maketitle

\section{Introduction}
We show a small but non-trivial example of using Alectryon
for writing a documentation (in pdf) of a proof written in Coq.

\section{Preliminaries}
We prove that, if a type $A$ is well founded for a relation \textit{<},
then the set of sorted (w.r.t. the relation $\geq$ ) sequences on $A$  is well-founded too.
Such sequences are represented in a compact way as multisets over $A$. 
For instance, $\langle a,a,a,b,c,c,c,c,c \rangle $ where $a>b>c$ is represented by the list \texttt{(a,2)::(b,0)::(c,4)::nil}.


Note that a factor $(a,n)$ is interpreted as $n+1$ copies of $a$.

\inputsnippets{MultisetWf/tDef}

Let us consider a relation over some type $A$. We define a lexicographic order on ($\texttt{t}\;A$).

\inputsnippets{MultisetWf/AGiven, MultisetWf/lexpowerDef}

\section{\texttt{lexpower}  is not well-founded in general}

Let us build an infinite strictly decreasing (w.r.t. \texttt{lexpower lt}) sequence.


\inputsnippets{MultisetWf/notWfa}

Then, we prove that any element of the sequence \texttt{seq} is non-accessible.

\inputsnippets{MultisetWf/notWfb}

It's clearly a contradiction, since (\texttt{seq 0}) is an element of the sequence, and accessible (because of the hypothesis \texttt{Hwf}).

\inputsnippets{MultisetWf/notWfc}
\section{Lists in normal form}


We say that a list $(a_1,n_1)::(a_2,n_2)::\dots::(a_p,n_p)::nil$  in  (\texttt{t A}) is in \emph{normal form} if  for any $1<=i<n$,
we have $\texttt{ltA}\; a_{i+1}\; a_i$.

\inputsnippets{MultisetWf/lexnfDef}

The \emph{lexicographic exponentiation} of \texttt{ltA} is the restriction
of \texttt{lexpower ltA} to the set of lists in normal form.
In the next section, we prove that, if \texttt{ltA} is well-founded, then its lexicographic power is well-founded too.
\inputsnippets{MultisetWf/lexltDef}

\paragraph{Note:} The restriction of a binary relation to a set is defined in the file \texttt{Restriction.v}.

\inputsnippets{Restriction/restrictionDef}


\section{Proof of well-foundedness}
It's time to prove that the lexicographic power of a well-founded relation is well-founded too. Let us open a work section.

\inputsnippets{MultisetWf/bigProofa}

We want to prove the following statement:

\inputsnippets{MultisetWf/theStatement}



\subsection{A first attempt}
We may try to prove \texttt{lexwf} by (structural) induction on \texttt{l}.

\inputsnippets{MultisetWf/BadProof,
  MultisetWf/BadProofb}

By the hypothesis \texttt{H0}, the list \texttt{y} may be
equal to \texttt{(a,n)::l'}, where \texttt{LT l' l}. But the induction hypothesis \texttt{IHl} is useless for solving the current goal.
\inputsnippets{MultisetWf/BadProofc}

This is a classical issue. In a naive proof by induction, some variables are fixed too early: here, the variable \texttt{l} in \texttt{IHl}.

\subsection{Introducing a new predicate}

We can express the well-foundedness of \texttt{LT} in the following way :
\begin{quote}
  \begin{enumerate}
    \item  ``The empty list is accessible''
  \item   `` For any $a$ in $A$ and $n$ in \texttt{nat},  any well-formed list starting with $(a,n)$ is accessible. ''

  \end{enumerate}
 \end{quote}

The first property is easy to prove by inversion.
 
\inputsnippets{MultisetWf/AccNil}

For the second one, we define a new predicate over $A$.
 
\inputsnippets{MultisetWf/AccsDef}

The new goal is to prove that any element of $A$ satisfies
the predicate \texttt{Accs}.


\subsection{Inversion lemmas}

Before starting the proof, we  prove three useful inversion lemmas:

\inputsnippets{MultisetWf/NFInv1}
\inputsnippets{MultisetWf/NFInv2}
\inputsnippets{MultisetWf/LTInv}

\subsection{The proof by induction}
Le us prove, by induction on $a:A$ that every element of $A$ satisfies \texttt{Accs}.

\inputsnippets{MultisetWf/LAccsa}

Let us note that, for any list \texttt{(a,n)::l} in normal form,
the tail \texttt{l} is either empty or starts with some pair \texttt{(b,p)} where \texttt{ltA b a}, hence is accessible (thanks to \texttt{IHa}).

\inputsnippets{MultisetWf/LAccsb}
\inputsnippets{MultisetWf/LAccsg}

The current goal is universally quantified over \texttt{n}.
Thus we may prove it by well founded induction over \texttt{nat}. 

\inputsnippets{MultisetWf/LAccsc, MultisetWf/LAccsd}

Thanks to \texttt{Hl}, we know that \texttt{l} is accessible.
Thus we can build an induction over \texttt{l}'s accessibility.




\inputsnippets{MultisetWf/LAccse}

In order to prove the accessibility of the list \texttt{((a, n) :: x0)}, we have to prove that any predecessor $y$ of this list is accessible, and consider every case (thanks to \texttt{LT\_inv}).

\inputsnippets{MultisetWf/LAccsf}

\subsubsection{First case}
If \texttt{y=nil}, then \texttt{y} is clearly accessible. 


\inputsnippets{MultisetWf/case1}

\subsubsection{Second case}
Let us consider a well formed list \texttt{y} of the form \texttt{(b,p)::l''}, where \texttt{ltA b a}. By \texttt{IHa}, this list is accessible.

\inputsnippets{MultisetWf/case2}

\subsubsection{Third case}
The list \texttt{y} may be of the form  \texttt{(a,n)::l''}, where \texttt{LT l'' l}. By \texttt{H2},  \texttt{y} is accessible.

\inputsnippets{MultisetWf/case3}
  
\subsubsection{Last case}
The list \texttt{y} may be of the form  \texttt{(a,p)::l''}, where \texttt{p<n}. Thus, we may apply \texttt{Hn}.


\inputsnippets{MultisetWf/case4}

\subsection{At last!}


\inputsnippets{MultisetWf/NFAcc}


\inputsnippets{MultisetWf/lexwf}




% \section{Examples}

% \inputsnippets{MultisetWf/Examples}

\section{Impossibility proofs}
There is no function of type \texttt{t nat ->  nat} which could serve as a measure for proving \texttt{lexlt lt}'s well-foundedness.


\inputsnippets{MultisetWf/Impossibility1}
\inputsnippets{MultisetWf/Impossibility1a}
\inputsnippets{MultisetWf/Impossibility1b}

\emph{Same exercise with \texttt{nat*nat} (with the lexicographic square of \texttt{lt})}.

\end{document}
