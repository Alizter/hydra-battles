\chapter{Gaia and the hydras (draft)}

\section{Introduction}


\section{Data refinement}

\emph{The following definitions and lemmas are in
\href{../theories/html/gaia_hydras.T1Bridge.html}%
{\texttt{gaia\_hydras.T1Bridge}}}.

Since both libraries \texttt{gaia} and \texttt{hydra-battles} have their own type \texttt{T1}, we will work with the following notations.

\inputsnippets{T1Bridge/hT1gT1}

Most used constants and functions will also get standardized
non-ambiguous names.


\inputsnippets{T1Bridge/MoreNotations}



The bridge between both libraries is made of two functions.

\inputsnippets{T1Bridge/iotaPiDef}

\inputsnippets{T1Bridge/iotaPiRw, T1Bridge/piIotaRw}

The functions \texttt{iota} and \texttt{pi} allow us to define
a notion of refinement for constants, functions, predicates and relations.

\inputsnippets{T1Bridge/refineDefs}

\subsection{Examples of refinements}
Both libraries define constants like $0$, $1$, $\omega$, and arithmetic functions: successor, addition, multiplication, and exponential of base $\omega$ (function $\phi_0$). We prove consistency of these definitions. Finally, we prove that the boolean predicates `` to be in normal form'' are equivalent.

\subsubsection{A few constants}
For each constant: $0$, $1$, \dots $n$,  $\omega$ we prove
that the \texttt{hydras}' definition refines \texttt{gaia}'s 
\inputsnippets{T1Bridge/constantRefs}

\subsubsection{Unary functions}

\inputsnippets{T1Bridge/unaryRefs}
