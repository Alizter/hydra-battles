\chapter{Gaia and the hydras (draft)}

\section{Introduction}


\section{Data refinement}

\emph{The following definitions and lemmas are in
\href{../theories/html/gaia_hydras.T1Bridge.html}%
{\texttt{gaia\_hydras.T1Bridge}}}.

Since both libraries \texttt{gaia} and \texttt{hydra-battles} have their own type \texttt{T1}, we will work with the following notations.

\inputsnippets{T1Bridge/hT1gT1}

Most used constants and functions will also get standardized
non-ambiguous names.


\inputsnippets{T1Bridge/MoreNotations}



The bridge between both libraries is made of two functions.

\inputsnippets{T1Bridge/iotaPiDef}

\inputsnippets{T1Bridge/iotaPiRw, T1Bridge/piIotaRw}

The functions \texttt{iota} and \texttt{pi} allow us to define
a notion of refinement for constants, functions, predicates and relations.

\inputsnippets{T1Bridge/refineDefs}

\subsection{Examples of refinements}
Both libraries define constants like $0$, $1$, $\omega$, and arithmetic functions: successor, addition, multiplication, and exponential of base $\omega$ (function $\phi_0$). We prove consistency of these definitions. Finally, we prove that the boolean predicates `` to be in normal form'' are equivalent.

\subsubsection{A few constants}
For each constant: $0$, $1$, \dots, $n$ and $\omega$ we prove
that the \texttt{hydras}' definition refines \texttt{gaia}'s.
\inputsnippets{T1Bridge/constantRefs}

\subsubsection{Unary functions}

\inputsnippets{T1Bridge/unaryRefs}

\subsubsection{Binary operations}
The binary operations $+$ and $\times$ are defined the same way in both libraries. Please note that they require comparison of ordinal terms. So, the proof of the following lemmas applies
compatibility lemmas like \texttt{compare\_ref} (see Section~\vref{sect:gaia-compare-ref}).

\inputsnippets{T1Bridge/plusRef, T1Bridge/multRef}

\subsubsection{Porting a theorem}
The \texttt{Gaia} library containsa proof that the multiplication of ordinals less than $\epsilon_0$ is associative.

\emph{From~\href{https://github.com/coq-community/gaia/blob/master/theories/ssete9.v}{gaia.ssete9.v}}

\begin{Coqsrc}
Lemma mulA: associative T1mul.
\end{Coqsrc}

This lemma was missing from \texttt{hydra-battles}. Nevertheless, we could adapt this lemma to \texttt{hydra-battles}' ordinals, by a small sequence of rewritings.

\inputsnippets{T1Bridge/multA}

The module~\href{../theories/html/gaia_hydras.T1Bridge_use.html}%
{\texttt{gaia\_hydras.T1Bridge\_use}} shows an example of
application of \texttt{multA}.

\inputsnippets{T1Bridge_use/T1BridgeUse}

\subsection{Order and comparison}
Leibniz equality and strict order on both libraries are
compatible:
\inputsnippets{T1Bridge/eqRef}

\label{sect:gaia-compare-ref}
\inputsnippets{T1Bridge/compareRef}

\inputsnippets{T1Bridge/ltRef}

\inputsnippets{T1Bridge/decideHLtRw}

Hence, both definitions of ``being in Cantor normal form'' are
compatible.

\inputsnippets{T1Bridge/nfRef}