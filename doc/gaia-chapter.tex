\chapter{Gaia and the hydras (experimental)}
\index{coq}{Plug-ins!Gaia}
\label{gaia-chapter}

\section{Introduction}
The \gaia project~\cite{Gaia} by Jos\'e Grimm aimed to formalize mathematics in \coq  in
the style of Nicolas Bourbaki. The formalization of the first book in the
Elements of Mathematics series by Bourbaki, on the theory of sets, was
initially described in a technical report in July 2009~\cite{Grimm2009a}.
The set-theoretic axioms and basic definitions in \gaia were derived
from an earlier development by Carlos Simpson~\cite{Simpson2004,CatsZFCContrib}.
Grimm then wrote (and continually updated) technical reports describing the
formalization of Bourbaki's two subsequent books~\cite{Grimm2009b,Grimm2016}
and additional topics in number theory~\cite{grimm:hal-00911710,Grimm2014},
before he passed away in 2019.

In 2020, members of \community transferred the \gaia source code to
GitHub and adapted it for recent releases of the Mathematical Components
library, which \gaia heavily relies on.
Anonymous volunteers (``collaborators of Nicolas Bourbaki'') then finished
the only in-progress proof left by Grimm. At around 155,000 LOC, \gaia is currently one of the largest maintained open source \coq projects.

\gaia contains definitions of ordinals in Cantor and Veblen normal form~\cite{grimm:hal-00911710}, adapted from the historical Cantor contribution~\cite{CantorContrib}. The data types for ordinals are essentially defined the same way as in \Hydras, but they are not identical inside \coq, e.g., due to residing in different modules. There are also minor differences in how ordinal arithmetic is implemented, due to the different evolutionary paths taken since divergence from the ancestor library.



The bridge we aim to build and cross is a two-way bridge:

\begin{itemize}
\item \gaia contains lemmas on ordinal arithmetic (\emph{e.g.} associativity of multiplication of ordinals strictly less than $\epsilon_0$), which were missing from \Hydras. 
  \item The other way, \Hydras contains formal definitions and proofs of properties of canonical sequences and rapidly growing functions  (see Chapters~\ref{chap:ketonen} and~\ref{chap:alpha-large}~of this book).
  \end{itemize}
  
  We plan to make accessible the first kind of lemmas to \HydrasLib' users and the second kind to \gaia's. For this purpose, we write modules dedicated to the importation of definitions and lemmas from each of both libraries into the other one.
Please note that by ``import'' we do not mean ``duplicate the proofs''.  Whenever possible, we proceed by rewriting steps over the statements and not the proofs. 



This initial draft bridge code mostly uses the SSReflect proof language
and idioms from the Mathematical Components library. We made this design
decision since we believe it is less challenging to reason about
Hydra-battles code using SSReflect and MathComp than to reason about
Gaia without SSReflect \footnote{The author of these proof scripts is still a beginner in Ssreflect. Please forgive the clumsiness of the current proof scripts.}.



\section{Library structure}
The \gaiaHydras library is designed for use  by users of
\gaia and/or  \HydrasLib. Whenever possible, we managed to respect the notations and naming conventions of both libraries.

In case of name clash (for instance the type \texttt{T1} of ordinal terms below $\epsilon_0$), we often added a one-letter prefix to names : \texttt{g} or \texttt{G} for \gaia, \texttt{h} or \texttt{H}
for \HydrasLib.

In the following two examples, the bare name \texttt{T1} is given by default either  to \HydrasLib' world or \gaia's. 
 

  
\begin{itemize}
\item In
  \href{../theories/html/gaia_hydras.GaiaToHydra.html}%
  {\texttt{gaia\_hydras.GaiaToHydra}}, we adopt \HydrasLib's point of view, and the name \texttt{T1} is by default bound to
\HydrasLib' data structures.

\inputsnippets{GaiaToHydra/T1BridgeUse,
  GaiaToHydra/LocateT1}

Associativity of multiplication over \HydrasLib' type \texttt{T1},
although imported from \gaia, is expressed with \HydrasLib' notations.

\inputsnippets{GaiaToHydra/T1BridgeUseb}

\item The other way, the module \href{../theories/html/gaia_hydras.GCanon.html}%
  {\texttt{gaia\_hydras.GCanon}} is an adaptation to \gaia of
  \href{../theories/html/hydras.Epsilon0/Canon.html}%
  {\texttt{hydras.Epsilon0/Canon}}.
  In this module, the name \texttt{T1} is bound by default to \gaia's meaning. \HydrasLib' name \texttt{T1} is renamed into \texttt{hT1}.

  The following lemmas are expressed in \gaia's vocabulary.
  \inputsnippets{GCanon/gcanonLimitMono,
    GCanon/gcanonLimitOf}
  
\end{itemize}

The \gaiaHydras entry point is the file
\href{../theories/html/gaia_hydras.T1Bridge.html}%
{\texttt{theories/gaia.T1Bridge.v}}, which requires modules both from \HydrasLib and \gaia. By default, we give priority to \gaia's notation and vocabulary.


\inputsnippets{T1Bridge/Requirements}

\inputsnippets{T1Bridge/hT1gT1}



\subsection{Data refinement}

\subsubsection{Renamings}

Most used constants and functions are also unambiguously renamed.


\inputsnippets{T1Bridge/MoreNotations}

Please note that in this module, we favour \gaia's notations over hydra's. A snippet like \texttt{(alpha < succ beta)} should be interpreted as \texttt{(T1lt alpha beta)}, where \texttt{T1lt} is defined in ~\href{https://github.com/coq-community/gaia/blob/master/theories/ssete9.v}{\texttt{gaia.ssete9.v}}.


\subsubsection{Translation functions and data refinement}

The bridge between both libraries is made of two functions.

\inputsnippets{T1Bridge/h2gG2hDef}

The following cancel lemmas will be often used in order to simplify sub-terms of the form (\texttt{g2h (h2g {\it t})}) and (\texttt{h2g (g2h {\it t})}) which appear in many proofs by rewriting.

\inputsnippets{T1Bridge/h2gG2hRw, T1Bridge/g2hH2gRw}

\inputsnippets{T1Bridge/h2gEqIff, T1Bridge/g2hEqIff}



Functions \texttt{h2g} and \texttt{g2h} allow us to define
a notion of ``data-refinement''  for constants, functions, predicates and relations. The following definitions express that some
constant, function, relation defined in \HydrasLib ``implements'' the same concept of \gaia.

\inputsnippets{T1Bridge/refineDefs}

\subsection{Examples of refinement}
Both of our libraries define constants like $0$, $1$, $\omega$, and arithmetic functions: successor, addition, multiplication, and exponential of base $\omega$ (function $\phi_0$). We prove that these definitions are mutually consistent. Finally, we prove that the boolean predicates `` to be in normal form'' are equivalent.

\subsubsection{A few constants}
For each constant: $0$, $1$, \dots, $n$ and $\omega$, we prove
that \texttt{hydras}' definition refines \texttt{gaia}'s.
\inputsnippets{T1Bridge/constantRefs}

%\inputsnippets{T1Bridge/utf8try}

\subsubsection{Unary functions}

\inputsnippets{T1Bridge/unaryRefs}

\subsubsection{Order and comparison}
Leibniz equality and strict order on both libraries are
compatible:


\inputsnippets{T1Bridge/eqRef}

\label{sect:gaia-compare-ref}
\inputsnippets{T1Bridge/compareRef}

\inputsnippets{T1Bridge/ltRef, T1Bridge/leRef}

\inputsnippets{T1Bridge/decideHLtRw}

\subsubsection{Cantor normal form}

Hence, both definitions of ``being in Cantor normal form'' are
compatible.

\inputsnippets{T1Bridge/nfRef}


\subsubsection{Addition and multiplication}

The binary operations $+$ and $\times$ are defined the same way in both libraries. Please note that they require comparison of ordinal terms. So, the proof of the following lemmas applies
compatibility lemmas like \texttt{compare\_ref} (see Section~\vref{sect:gaia-compare-ref}).

\inputsnippets{T1Bridge/plusRef, T1Bridge/multRef}

\subsection{Looking for a lemma}
\coq's command \texttt{Search} and notation scopes allow you to explore both libraries.

\inputsnippets{T1Bridge/SearchDemo}

\section{Importing a theorem from \gaia}
The \texttt{Gaia} library already contains many lemmas about
ordinal arithmetic. In this section, we give two examples of
porting such a lemma to \HydrasLib' vocabulary.

\subsection{Associativity of ordinal multiplication (below \texorpdfstring{$\epsilon_0$}{epsilon\_0})}
\gaia already contains a proof that the multiplication of ordinals less than $\epsilon_0$ is associative.
\emph{From~\href{https://github.com/coq-community/gaia/blob/master/theories/ssete9.v}{gaia.ssete9.v}}

\begin{Coqsrc}
Lemma mulA: associative T1mul.
\end{Coqsrc}

This lemma was missing from \texttt{hydra-battles}. Nevertheless, we could adapt this lemma to \texttt{hydra-battles}' ordinals, by a small sequence of rewritings.

\inputsnippets{T1Bridge/multA}

The module~\href{../theories/html/gaia_hydras.GaiaToHydra.html}%
{\texttt{gaia\_hydras.GaiaToHydra}} shows a small
example of importation of \texttt{multA} into \texttt{hydra-battles}' world.

\inputsnippets{GaiaToHydra/T1BridgeUse}

In the rest of this module, names like  \texttt{T1}, \texttt{omega}, etc. are  bound to \HydrasLib' meaning.

 \inputsnippets{GaiaToHydra/LocateT1, GaiaToHydra/T1BridgeUseb}

 \subsection{Right Distributivity, etc.}
 Likewise, we prove almost for free that ordinal multiplication is right distributive over addition (with \HydrasLib' definitions).

 \inputsnippets{GaiaToHydra/distributivity}

We plan to automatize as soon as possible this kind of transfer of lemmas from \gaia to \HydrasLib.


\section{Importing theorems from \HydrasLib}


Many lemmas of the library \texttt{ordinals/Epsilon0/} are statements about inhabitants of type \texttt{hT1}. In order to get
similar statements on \gaia's type \texttt{T1}, we prove and
apply rewriting rules of the following forms:

\vspace{4pt}

\noindent\emph{From Module~\href{../theories/html/gaia.T1Bridge.html}{gaia.T1Bridge}}

\inputsnippets{T1Bridge/T1leIff}
\inputsnippets{T1Bridge/rewritingRules}
\inputsnippets{T1Bridge/T1ltIff}


\inputsnippets{T1Bridge/limitbRef, T1Bridge/isSuccRef}


\subsection{A type for well-formed ordinal terms}

For sake of compatibility, we add a clone of \texttt{hydra-battles}' type \texttt{E0} defined in \href{../theories/html/hydras.Epsilon0.E0.html}{Epsilon0.E0}.

\inputsnippets{T1Bridge/E0Def}
\inputsnippets{T1Bridge/E0plusMultDef}


\inputsnippets{T1Bridge/E0EqP}

In order to import definitions and lemmas
from~\href{../theories/html/hydras.Epsilon0.E0.html}{Epsilon0.E0}, we define a pair of translations.

\inputsnippets{T1Bridge/gE0h2gG2h}

These tools allow us to import \texttt{E0.lt}'s  well-foundedness.

\inputsnippets{T1Bridge/gE0LtWf}

\subsection{E0 as an ordinal notation}
The module \href{../theories/html/gaia.T1Bridge.html}{gaia.T1Bridge} defines an instance of class \texttt{ON E0lt compare}
(see Chapter~\ref{chap:ON}).

\inputsnippets{T1Bridge/T1compareDef}
\inputsnippets{ T1Bridge/T1compareCorrect}
\inputsnippets{T1Bridge/E0compare}
\inputsnippets{T1Bridge/E0Sto}
\inputsnippets{T1Bridge/gEpsilon0}

\inputsnippets{T1Bridge/ExampleComp}

\subsection{Canonical sequences}


In~\href{../theories/html/gaia_hydras.GCanon.html}%
{\texttt{gaia\_hydras.GCanon}}, canonical sequences are just imported from the library \linebreak
\href{../theories/html/hydras.Epsilon0.Canon.html}%
{\texttt{hydras.Epsilon0.Canon}}.
First, we rename \HydrasLib' \texttt{canon} function into
\texttt{hcanon}. Then we define a new function \emph{via} the
translations \texttt{g2h} and \texttt{h2g}.

\inputsnippets{GCanon/canonDef}

The following lemmas are proved by rewriting from corresponding statements proved in \HydrasLib.


\inputsnippets{GCanon/g2hCanonRw}
\inputsnippets{GCanon/gcanonLt}
\inputsnippets{GCanon/glimitCanonSNotZero}
\inputsnippets{GCanon/gcanonLimitMono}

Let us recall \gaia's definition of $\omega$-limit.

\inputsnippets{nfwfgaia/LimitV2Def, nfwfgaia/LimitOFDef}

We adapt theorems from \href{../theories/html/hydras.Epsilon0.Canon.html}%
{\texttt{hydras.Epsilon0.Canon}} to \gaia's vocabulary.

\inputsnippets{GCanon/gcanonLimitStrong}
\inputsnippets{GCanon/gcanonLimitV2}
\inputsnippets{GCanon/gcanonLimitOf}

\subsection{A family of rapidly growing functions}

The functions $F_\alpha$ are described in Section~\vref{sect:wainer}. The symbol \texttt{F\_} has type \texttt{E0 -> nat -> nat},
where \texttt{E0} is the type of well formed ordinal terms less than $\epsilon_0$.

\subsubsection{Importing the \texttt{F\_alpha} family}

We are now able to import \HydrasLib' $F_\alpha$ functions.

\inputsnippets{GF_alpha/FAlphaDef}





The following definitions adapt \HydrasLib' to \ssreflect's 
definitions of $<$ and $\leq$ on \texttt{nat}.

\inputsnippets{T1Bridge/MonoDef}

By rewriting, we import a few lemmas from
~\href{../theories/html/hydras.Epsilon0.F_alpha.html}{Epsilon0.F\_alpha}.



\inputsnippets{GF_alpha/FAlphaProps}

With the same technique, we prove that if
$\alpha\geq\omega$, then $F_\alpha$ is not primitive recursive.

\inputsnippets{GF_alpha/FAlphaNotPR}






% \section{Gaia's proof of well-foundedness}
% The library \HydrasLib already contains two proofs
% of well-foundedness of the strict order on ordinal terms in
% Cantor normal form: a direct proof~\vref{sec:T1Wf} and a proof based on the recursive path ordering~\vref{remark:a3pat}, by
% \'Evelyne Contejean. We found it interesting to comment Gaia's proof, which uses tools already present in~\cite{CantorContrib}.



% This proof is quite dense and qualified as ``a bit tricky''. We just present here its main steps with the help of the \alectr tool.

% The reader is invited to replay the full proof in  \href{https://github.com/coq-community/gaia/blob/master/theories/ssete9.v}{gaia.ssete9.v}, and to compare with
% the proof in Section~\ref{sec:strongly-accessible}.

% \begin{remark}
%   The following snippets have been generated, not from
%   \href{https://github.com/coq-community/gaia/blob/master/theories/ssete9.v}{gaia.ssete9.v}, but on a provisional copy:
% \url{../theories/gaia/nfgaia.v} we made in order to insert comments for \alectr.  This copy is planned to be removed when we are able to add \alectr directives to a \coq source we are not allowed to modify.
% \end{remark}

% \subsection{Restricted recursion}
% As remarked in~\vref{sec:T1Wf}, the order \texttt{T1lt} on \texttt{T1} is not well-founded, because of terms not in normal form.
% But the \emph{restriction} of \texttt{T1lt} to terms in normal form \emph{is} well-founded. The following section introduces a  vocabulary in order to reason about statements of the form  ``the restriction of the relation $R$ to a subset $P$ of $A$ is well founded''.




% \inputsnippets{nfwfgaia/restrictedRecursion}

% The following induction principle is just a variant of \texttt{well\_founded\_induction\_type} for relations defined with \texttt{restrict}.

% \inputsnippets{nfwfgaia/restrictedRecursiona}

% With $P=\texttt{T1nf}$, this principle will allow us to write proofs by transfinite induction.

% \subsection{Main proof steps}

% \emph{Notes: the abbreviation \texttt{LT}, and the \texttt{fold LT} and \texttt{unfold LT} tactics,are not in \texttt{gaia}, but  have been added to make the goal display more readable.
% The infix notation \texttt{`` \_ < \_''}  is bound to the order \texttt{T1lt} in \gaia, while,  in \HydrasLib,  it is bound to its restriction \texttt{LT} to terms in normal form. }

% \inputsnippets{nfwfgaia/nfWfProofa}
% We start with a classic structural induction on the type \texttt{T1}.

% \inputsnippets{nfwfgaia/nfWfProofaa}
% The following \texttt{ssreflect} tactic  pushes a new hypothesis \texttt{Hb} which --- in the same terms as in Section~\vref{sec:strongly-accessible} --- tells that any ordinal less than $a$ is ``strongly accessible''.

% \inputsnippets{nfwfgaia/nfWfProofb}

% The following sentence gives us the possibility of induct over \texttt{a}'s accessibility.

% \inputsnippets{nfwfgaia/nfWfProofbb}


% By the following two forward steps, we show that $b$ is accessible, and will be able to prove properties by induction on $b$'s accessibility.

% \inputsnippets{nfwfgaia/haveHc}

% \inputsnippets{nfwfgaia/haveHd}

% \inputsnippets{nfwfgaia/hdProved}

% Now, let us consider any ordinal term which is (by \texttt{LT}) less than \texttt{(cons a n b)}.

% There are three cases to consider.
% \begin{enumerate}
% \item It is of the form \texttt{(cons a' n' b')} where \texttt{a'} is strictly less than \texttt{a}.
%   \inputsnippets{nfwfgaia/nfWfProofd}
  
%   Then, it is accessible (by hypothesis \texttt{Hb}).
  
%   \item The considered term  is equal to
%     \texttt{(cons a n b'')}, where $b''<\omega^a$. 
%     \inputsnippets{nfwfgaia/nfWfProofg}
    
%     Then we can replace \texttt{a''} with \texttt{a}, \texttt{n''} with \texttt{n}, and apply the hypotheses \texttt{Hd} and \texttt{He}.
    
% \item
%   The term is equal to
%   \texttt{(cons a n'' b'')}, where $n''<n$ and $b''<\omega^a$.

  
  
%   \inputsnippets{nfwfgaia/nfWfProofe}

%   By \texttt{Hd} and \texttt{He}, and a few technical lemmas about normal forms,   the term  \texttt{(cons a n b'')} is accessible, and so is \texttt{(cons a n'' b'')}.

 
% \end{enumerate}

% In every case, the considered ordinal is accessible. By the definition of accessibility, the ordinal \texttt{(cons a n b)} is accessible, which ends the proof. 


