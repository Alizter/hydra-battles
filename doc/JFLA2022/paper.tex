% easychair.tex,v 3.5 2017/03/15

\documentclass{easychair}
%\documentclass[EPiC]{easychair}
%\documentclass[EPiCempty]{easychair}
%\documentclass[debug]{easychair}
%\documentclass[verbose]{easychair}
%\documentclass[notimes]{easychair}
%\documentclass[withtimes]{easychair}
%\documentclass[a4paper]{easychair}
%\documentclass[letterpaper]{easychair}

\usepackage{doc}
\usepackage{tikz}
\usepackage{tikzsymbols}
\usepackage{pifont}
\newtheorem{theorem}{Theorem}
\usetikzlibrary{arrows}

%\usepackage[firstpageonly=false, color={[gray]{0.5}},
%   scale=2.0, text=DRAFT]{draftwatermark}
% -------------------------------
%%%% For Alectryon

\usepackage{texments}
%%% for movies by alectryon
\usepackage{./assets/alectryon}
\usepackage{./assets/pygments}
%%% One hypothesis per line 
\makeatletter
\renewcommand{\alectryon@hyps@sep}{\alectryon@nl}
\makeatother

%%% \snippets{A,B,C,…} inputs a series of snippets as one block (with \itemsep
%%% between them).  A, B, C should be paths to files in snippets/.
\usepackage{etoolbox}
\makeatletter
\newcommand{\inputsnippets}[1]
  {{\setlength{\itemsep}{1pt}\setlength{\parsep}{0pt}% Adjust spacing
    \alectryon@copymacros\begin{io}
      \forcsvlist{\item\@inputsnippet}{#1}
    \end{io}}}
\let\input@old\input % Save definition of \input
\newcommand{\@inputsnippet}[1]
  {{\renewenvironment{alectryon}{}{}% Skip \begin{alectryon} included in snippet
    \input@old{snippets/#1}}}
\makeatother

%---------------------------- 
\newcommand{\canonseq}[2]{\mbox{$\{#1\}(#2)$}}

\usepackage{varioref}
\newtheorem{todo}{To do}
\usepackage{amsfonts, afterpage}
\usepackage{xspace}

\newcommand{\easychair}{\textsf{easychair}}
\newcommand{\miktex}{MiK{\TeX}}
\newcommand{\texniccenter}{{\TeX}nicCenter}
\newcommand{\makefile}{\texttt{Makefile}}
\newcommand{\latexeditor}{LEd}
\newcommand{\Hydras}{Hydras \& Co.\xspace}
\newcommand{\Coqcommunity}{Coq-community\xspace}


%\makeindex

%% Front Matter
%%
% Regular title as in the article class.
%
\title{Hydras, Ordinals \&  Co.  \\
  A library in Coq of entertaining formal mathematics}



\author{
Pierre Castéran \inst{1}
\and
    Jérémy Damour \inst{2}
\and
Karl Palmskog \inst{3}
\and Clément Pit-Claudel \inst{4}
\and Théo Zimmermann \inst{5}
}


\institute{
Univ. Bordeaux, CNRS, Bordeaux INP, LaBRI, UMR 5800, F-33400 Talence, France \\
  \email{pierre.casteran@labri.fr}
\and
Univ. de Paris, F-75013 Paris, France
\and
KTH Royal Institute of Technology, Stockholm, Sweden
\and
MIT CSAIL, Cambridge, Massachusetts, USA
\and
Inria, Univ. de Paris, CNRS, IRIF, UMR 8243, F-75013 Paris, France
}



\authorrunning{Castéran, Damour, Palmskog, Pit-Claudel and Zimmermann}

\titlerunning{Hydras, Ordinals \& Co.}

\newcommand{\TODO}[2][]{[\textcolor{red}{TODO (#1):} \emph{#2}]}
\newcommand{\coq}{Coq\xspace}
\newcommand{\community}{Coq-community\xspace}
\newcommand{\gaia}{Gaia\xspace}
\newcommand{\alectr}{Alectryon\xspace}
\newcommand{\equations}{Equations\xspace}
\newcommand{\Hydras}{Hydras \& Co.\xspace}

\begin{document}

\maketitle


\begin{abstract}
  \Hydras is a collaborative library on discrete math, written for the \coq  proof assistant, and developed under the \community project. The proof scripts are
  accompanied with an electronic book, made with the help of the \alectr litterate proving tool.
  We present the evolution of the mathematical contents since
  former presentations at JFLA meetings.
  Then, we show how the structure of the project is determined   by two  requirements which must be continuously satisfied:
   maintenance of the library with respect to several \community projects it requires and evolutions of \coq, and 
  consistency of the book with respect to the library. 
\end{abstract}


% \setcounter{tocdepth}{2}
% {\small
% \tableofcontents}


%------------------------------------------------------------------------------
\section{Introduction}
\label{sect:introduction}

\subsection{Background}

\begin{itemize}
\item formalizations of mathematics in proof assistants, such as \coq, continually grow in size and scope
\begin{itemize}
\item the Mathematical Components library and the four-color theorem and the odd order theorem in \coq
\item the Mathlib library for Lean
\end{itemize}
\item to continue to be useful as examples and reusable libraries, formalizations must be well \emph{documented} and \emph{maintained} to ensure they work with new proof assistant releases. %\TODO[Pierre]{add evolution of styles of proof and formalization (not limited to Coq litterature)? } 
\item Lean and Mathematical Components have accompanying books, which are written mainly to onboard new library users
\item book authors faced with several design choices: literate documents, using tools to keep the book and the proof assistant synchronized
\end{itemize}

\subsection{Vision}
The \Hydras project, part of \community~\cite{CoqCommunity} aims to be an experimental platform for the collaborative development of commented libraries of formal proofs. \community is a community organization that we have founded in 2018 with two goals in mind: providing a solution for the long-term maintenance of interesting \coq packages, and working collaboratively on documentation projects. The \Hydras project demonstrates that these two goals are not independent: interesting \coq packages can become the basis for new documentation.
%
This umbrella project now includes evolved versions of the former Cantor and Additions libraries (under the new names of Hydra-battles and Addition-chains), the Ackermann sub-library, extracted from Russel O'Connor's Goedel library~\cite{OConnor05, Goedel} and a bridge to the \gaia library (by José Grimm~\cite{Gaia,grimm:hal-00911710}).
%
By following this approach of commenting interesting \coq packages, we provide new documentation content, that contributes to the diversity of the thriving \coq documentation ecosystem.

We call on the \coq users in the JFLA community and beyond to come and join us in this effort, by bringing new interesting projects which are worth presenting to \coq learners, \emph{a.k.a.} \coq users, and guiding them in their exploration.
%
We also always have project ideas to extend further our explorations and anyone is welcome to join the team by sending small or larger contributions through pull requests.
%
The current state of the project is already the result of such evolutions after several of us contributed project solutions and new proposals to the initial version of the first author.

Futhermore, contrary to traditionally published books, the ``book'' that forms part of this project is intended to be forever evolving. As new \coq formalization patterns and proof techniques appear, the book can be adapted to demonstrate their use (in case they fit well with our applications).
%
By using modern maintenance techniques such as continuous integration and deployment, we can ensure that this documentation stays up to date with the latest \coq releases. With \alectr~\cite{alectryonpaper, alectryongithub}, we ensure that code and documentation are always in sync.

\subsection{Hydra games}
Hydra games (also known as \emph{Hydra battles}) appear in an article published in 1982 by two mathematicians, 
Laurie Kirby and Jeff Paris: \emph{Accessible Independence Results for Peano Arithmetic}~\cite{KP82}.
This article describes a game between two players: Hercules and a hydra.
A short description of the game  can be found in~\cite{bauer2008, KP82, JFLA2018paper}. One can also play with
Andrej Bauer's simulator~\cite{BauerHydra}.
In a few words:
\begin{itemize}
\item A hydra is a finite tree, traditionally presented with the root at the bottom, the leaves of which are called \emph{heads}
  (Fig~\ref{fig:round}).
\item At every round, Hercules chops off one head of the hydra. If the head is at a distance greater than 1 from the root,
  then some sub-tree $h$ of the hydra is copied a certain amount $n$ of times. The number $n$ of copies and the sub-tree $h$ may depend of the considered variant of the game
  or the time elapsed since the beginning of the fight.
  Figure~\ref{fig:round} shows an example with $n=2$.
\end{itemize}



\begin{figure}[h]
  \centering
  \begin{tikzpicture}[very thick, scale=0.3]
  \node (h1) at (4,0){$\bullet$};
  \node[blue] (h2) at (4,2){$\bullet$};
  \node[blue] (h3) at (2,4){$\bullet$};
  \node[blue] (h4) at (0,6){$\bullet$};
  \node[blue] (h5) at (4,6){$\bullet$};
  \node[red!90!black] (h6) at (6,4){$\bullet$};
  \draw (h1) -- (h2) ;
  \draw (h2) -- (h3) ;
  \draw (h2) -- node[red!90!black,font=\small,sloped,shift={(0.01,-0.075)},rotate=90]{\textbf{\ding{34}}} (h6);
  \draw (h3) -- (h4) ;
  \draw (h3) -- (h5) ;
 
\node (hn1) at (14,0){$\bullet$};
\node[blue] (hn2) at (12,2) {$\bullet$};
\node[blue] (hn3) at (12,4) {$\bullet$};
\node[blue] (hn4) at (10,6){$\bullet$};
  \node[blue] (hn5) at (12,6){$\bullet$};
  \draw (hn1) -- (hn2) ;
  \draw (hn2) -- (hn3) ;
  \draw (hn3) -- (hn4) ;
  \draw (hn3) -- (hn5) ;
\node (hn2b) at (14,2) {$\bullet$};
\node (hn3b) at (14,4) {$\bullet$};
\node (hn4b) at (13.4,6){$\bullet$};
  \node (hn5b) at (14.6,6){$\bullet$};
  \draw (hn1) -- (hn2b) ;
  \draw (hn2b) -- (hn3b) ;
  \draw (hn3b) -- (hn4b) ;
  \draw (hn3b) -- (hn5b) ;
  \node (hn2c) at (16,2) {$\bullet$};
\node (hn3c) at (16,4) {$\bullet$};
\node (hn4c) at (16,6){$\bullet$};
  \node (hn5c) at (18,6){$\bullet$};
  \draw (hn1) -- (hn2c) ;
  \draw (hn2c) -- (hn3c) ;
  \draw (hn3c) -- (hn4c) ;
  \draw (hn3c) -- (hn5c) ;
\end{tikzpicture}

  \caption{Two successive states of a hydra in a battle.  Hercules chopped off the rightmost head of the hydra (red), and the whole left tree except the root node (blue) was copied twice.}
  \label{fig:round}
\end{figure}



 %  \begin{figure}[htb]
% \centering
% \begin{tikzpicture}[very thick, scale=0.3]
% \node (foot) at (10,0) {$\bullet$};
% \node (N1) at (2,2) {$\bullet$};
% \node (N2) at (10,2) {$\bullet$};
% \node (N22) at (7,2) {$\bullet$};
% \node (N3) at (14,2) {$\bullet$};
% \node (N4) at (18,2) {$\Smiley[2][green]$};
% \node (N5) at (0,4) {$\bullet$};
% \node (N6) at (2,5) {$\Smiley[2][green]$};
% \node (N7) at (4,6) {$\Smiley[2][green]$};
% \node (N88) at (7,4) {$\bullet$};
% \node (N8) at (10,4) {$\bullet$};
% \node (N9) at (14,6) {$\Smiley[2][green]$};
% \node (N10) at (0,8) {$\Smiley[2][green]$};
% \node (N11) at (10,7) {$\Smiley[2][green]$};
% \node (N111) at (7,7) {$\Smiley[2][green]$};
% \draw (foot) to [bend left=10] (N1);
% \draw (foot) -- (N2);
% \draw (foot) -- (N22);
% \draw (foot) -- (N3);
% \draw (foot) -- (N4);
% \draw (N1) to  (N5);
% \draw (N1) to   [bend left=10] (N6);
% \draw (N1) to   [bend right=20] (N7);
% \draw (N2) to  (N8);
% \draw (N22) to  (N88);
% \draw (N8) to  (N11);
% \draw (N88) to  (N111);
% \draw (N3) to  (N9);
% \draw (N5) to  (N10);
% \end{tikzpicture}
% \caption{The hydra associated with the ordinal $\omega^{\omega+2}+\omega^\omega \times 2 + \omega + 1$ \label{fig:iota-example}}

% \end{figure}

Kirby and Paris prove the following theorems, applying
combinatorial results about ordinal numbers by Jussi Ketonen and Robert Solovay~\cite{KS81}.

\begin{theorem}
  In the Hydra game, Hercules eventually wins, whichever the strategy of both players :
  choice of a head to chop off, choice of the number of copies. 
 \label{kp:thm1}
\end{theorem}

\begin{theorem}
  Theorem~\ref{kp:thm1} cannot be proved in Peano Arithmetic. \label{kp:thm2}
\end{theorem}

The contrast between the simplicity of the statements above and the complexity of their proofs convinced us that it is a good theme for a commented library~\cite{HydraBattles} of formal proofs written for the \coq proof assistant~\cite{Coq}. 

Complex formalisations and proofs are explained in an
  electronic book~\cite{HydraBook} (PDF document of over 280 pages). Whenever various reasonable choices exist, we try to present and compare the alternatives.
  For instance, Figures~\ref{fig:Ex42E0} and \ref{fig:Ex42-schutte} show two radically different proofs of the equality
  $\omega+42+\omega^2=\omega^2$. The first one is a simple proof by computation, the second one shows how this equality
  is a consequence of the axioms of the set-theoretic model  by Kurt Schütte~\cite{schutte}. 

This work is also an opportunity to 
 provide ``concrete'' examples of formalization and proof techniques: operational type classes, functions defined by  equations, dependently typed functions, etc. It may be also used as a library on ordinal numbers, for instance for proving termination properties.

 Prior stages of this project have already been presented at
 JFLA~\cite{PCiota, JFLA2018paper}.

We present recent evolutions of the library: new results, interaction within the \community project~\cite{CoqCommunity}, and documentation generated with \alectr~\cite{alectryonpaper, alectryongithub}.

\section{Changes in the library}
The 2018 article~\cite{JFLA2018paper} contains a formal proof of  a variant of Theorem~\ref{kp:thm2}:

\begin{theorem}
  Let $\mu$ be any ordinal strictly less than $\epsilon_0$.
  There is no function mapping hydras to the segment $[0,\mu)$ that could be used as a measure for  proving the termination of  all hydra battles.\label{thm3}
\end{theorem}

Considering measures applicable to \emph{all} battles allowed us to
focus on battles where the hydra can make an arbitrary number of copies at any time, which made our proof by contradiction artificially easier.
Unfortunately, the examples  most commonly shown in the litterature
(see for instance~\cite{KP82, bauer2008, BauerHydra}) 
assume that the hydra grows $n$ copies at step $n$ of the game, which is incompatible with our proof.

\vspace{6pt}

We prove now that Theorem~\ref{thm3} still holds with these typical battles by borrowing new combinatorial results from~\cite{KS81}.
Without loss of generality, we assume the following restrictions:
 
 \begin{itemize}
   \item The game starts at an initial step $i\in\mathbb{N}$ (not necessarily $0$).
   \item  The hydra is always the  representation as a tree of some ordinal strictly below $\epsilon_0$ in Cantor normal
     form. For instance, Fig~\ref{fig:round} shows the hydras respectively associated with  $\omega^{\omega^2+1}$ and $\omega^{\omega^2}\times 3$.
 
 \item Hercules always chops off the rightmost head of the hydra.
 \end{itemize}
 
 In mathematical terms, if at step $n$ the hydra is associated with the ordinal $\alpha$, at step $n+1$ it is associated with
 $\canonseq{\alpha}{n+1}$, the $(n+1)$th element of the canonical sequence of $\alpha$~\cite{KS81}.

 Our new proof of Theorem~\ref{thm3} is based on a systematic study of strictly decreasing sequences of ordinals below $\epsilon_0$, borrowed from~\cite{KS81}.
 
We also study the number of steps of a battle:
Let $\alpha<\epsilon_0$ be an ordinal. 
We prove that  the number of steps of the battle starting with
$\alpha$ at step $i$ is greater or equal than
$H'_\alpha(i)-i$, where $H'$ is a slight variant of the Hardy hierarchy of rapidly growing functions~\cite{BW85, KS81, Promel2013, Wainer1970}.  The function $H'_\alpha$ is defined by transfinite recursion over $\alpha$ on Figures~\ref{fig:hardy-math}
and~\ref{fig:Hprime}.


\begin{figure}[h]
\begin{align}
  H'_0(i) & = i\\
  H'_\alpha(i) &= H'_{(\canonseq{\alpha}{i+1})}(i)  \quad\textit{if $\alpha$ is a limit ordinal}\\
  H'_{\alpha}(i) &=H'_\beta(i+1) \quad\textit{if $\alpha=\beta+1$}
\end{align}  
  \caption{The $H'$ rapidly growing hierarchy of arithmetical functions}
  \label{fig:hardy-math}
\end{figure}


 \begin{figure}[h]
 \inputsnippets{Hprime/HprimeDef}
\caption{$H'$ definition with the \texttt{coq-equations}
 plug-in~\cite{sozeau:hal-01671777}}
\label{fig:Hprime}
\end{figure}


Using $H'$s equations as rewrite rules, we can study a realistic example. We take the hydra of figure~\ref{fig:start} and $i=0$ as initial configuration. 
By a sequence of rewritings and inductions, we prove that the number of steps of the considered battle is greater or equal than $2^{2^N}$, where $N=2^{70}-1$.

More generally, we prove that, for $\alpha\geq\omega^\omega$
the function computing the length of the battle starting with the configuration $(\alpha,i)$ is not primitive recursive.

\begin{figure}[h]
  \centering
  \begin{tikzpicture}[very thick, scale=0.3]
  \node (h1) at (4,0){$\bullet$};
  \node (h2) at (2,2){$\bullet$};
  \node (h3) at (0,4){$\bullet$};
  \node (h4) at (1,4){$\bullet$};
  \node (h5) at (2,4){$\bullet$};
  \node (h6) at (4.5,2){$\bullet$};
  \node (h7) at (6.5,2){$\bullet$};
  \node (h8) at (8.5,2){$\bullet$};
    \draw (h1) -- (h2) ;
    \draw (h2) -- (h3) ;
    \draw (h2) -- (h4) ;
    \draw (h2) -- (h5) ;
    \draw (h1) -- (h6) ;
    \draw (h1) -- (h7) ;
    \draw (h1) -- (h8);
 \end{tikzpicture}

  \caption{The hydra associated with the ordinal $\omega^3+3$}
  \label{fig:start}
\end{figure}




\label{sect:not-pr}

%\TODO[Clément]{Could we use a font with ligatures?}




\section{Integration within \Coqcommunity}

\subsection{Presentation}

\Coqcommunity is an informal organization run by volunteer Coq users with objectives to maintain interesting Coq projects in the long-term and help Coq users to collaborate on documentation, tooling, etc.

It was created in 2018, inspired by the {Elm Community} organization~\cite{zimmermann:tel-02451322}.
%
Such ``Community Package Maintenance Organizations'' actually exist in many ecosystems as they avoid the very common problem of an important package becoming unmaintained as its author has moved on or disappeared~\cite{zimmermann2021grounded}.

In the case of \coq, this problem may be even more prevalent as many packages are created by PhD students, or by researchers for a specific paper, but their authors do not intend to maintain the package in the long-term.
%
The authors are however generally very open to having someone else who expresses interest in their work continue the maintenance.
%

\Coqcommunity makes that easy by providing a process for transferring or forking an unmaintained package, tooling for setting up good maintenance practices (such as continuous integration) and by making it possible for someone to take over a package without committing in the long-term (as maintainers who drop out can easily be replaced by some other volunteers).

Today, \Coqcommunity hosts over 50 projects maintained by over 30 maintainers.
%
The hosted projects come from a variety of origins.
%
Some had been maintained in the past by the \coq development team on behalf of the authors, but this meant that only minimal changes required to make the project build with new \coq versions were conducted.
%
Some were still maintained by their authors but they wanted help to keep them up-to-date.
%
Others were simply left unmaintained and were revived by their adoption in \Coqcommunity.

\subsection{Interactions with other \Coqcommunity projects}

Given the objectives of \Coqcommunity, it makes sense to propose a transfer anytime we encounter an interesting project that is insufficiently maintained.
%
After the project is adopted, large changes and refactorings are explicitly allowed.
%
This means, for instance, that we can consolidate libraries together or split them into several packages if it makes sense.

In order to prove formally that the length of the considered
kind of battles is not given by a primitive recursive function, we used the formalization of primitive recursive functions, part
of Russell O'Connor's contribution on G\"{o}del first incompleteness theorem~\cite{OConnor05, Goedel}.
For this purpose, and above all by consideration of the scientific interest of this contribution, we chose to host and maintain this work within \Coqcommunity.

Computability is a key topic in Computer Science teaching. Moreover O'Connor's library is a nice illustration of dependently typed programming, so we chose to devote a full chapter to this formalization, with comments on the definitions and proofs, [counter]-examples and exercises, and  extract part of the Goedel library into a new sub-library of the \Hydras library: Ackermann.

% We could already talk a bit of the monorepo structure here, and
% defer some explanations on the tooling to the next section.

\section{Modernizing the infrastructure}

\subsection{Documentation with \alectr}


The \Hydras book is written in LaTeX, but it makes very frequent references (about once per page, 274 snippets over 281 pages) to parts of the \coq development, showing definitions (Fig.~\ref{fig:Hprime}), lemmas (Fig.~\ref{fig:Ex42E0}), proof scripts (Fig.~\ref{fig:Ex42-schutte}), and computation results (Fig.~\ref{fig:Pow-evalPow17LetIn}). The order in which these references appear in the book is independent of the structure of our libraries. Thus, we chose not to include our textbook as comments in the libraries (solution \emph{à la} Coqdoc).
%\TODO[Pierre]{is the preceding sentence OK?}

Originally, we copied snippets from \coq sources into LaTeX manually, and recorded and inserted the corresponding outputs manually as well.  This approach is common, but brittle: changes to \coq definitions or lemmas had to be reflected in the book's sources, and we found multiple instances where the book and the \coq development had diverged. %\TODO[Clément]{see FIXME cite alectryon for a discussion of various approaches to documentation of \coq developments}
 


We solved this maintenance issue by moving to \alectr, a tool to automatically record \coq proofs.  Instead of copy-pasting fragments into LaTeX, we now embed small LaTeX files automatically generated by \alectr from our \coq development.  Our build system guarantees that these LaTeX snippets are always up-to-date and consistent with the code (and, by comparing these snippets across releases of \coq or versions of \Hydras, we can easily spot unexpected changes).

The transition happened gradually, over a few weeks: for each snippet of \coq code that was in the book, we had to take the following steps:

\begin{enumerate}
\item Mark the snippet in the \coq sources (we use special comments \texttt{(* begin snippet \textit{name} *)} … \texttt{(* end snippet \textit{name} *)})
\item Configure output display, using special \alectr annotations to configure what should be shown (only the inputs, inputs and outputs, some steps of the proof but not all, some proof states at key moments in a proof, etc.)
\item Replace the copy-pasted inputs and output in LaTeX with an \texttt{input} command.
\end{enumerate}

Importing snippets into a LaTeX document was not one of the original use cases of \alectr, so we had to extend it: the original \alectr did not have support for exporting to LaTeX, and it was towards documenting individual source files (a style where code and prose are interleaved within the same file and code is documented in the same order as it is compiled).  We extended {alectr to support our needs by implementing a LaTeX backend, and by programming it to generate individual snippet files, one per \texttt{snippet} comment block.  The later part was straightforward: all it took was to build a custom \alectr \textit{driver}, a small (about 100 lines) Python program that leverages most of the \alectr toolchain but defines a custom frontend that understands our snippet annotations (and otherwise exposes the exact same command line as \alectr).

  \begin{figure}[h]
    \centering
    \fbox{
      \begin{minipage}[h]{1.0\linewidth}
        \inputsnippets{E0/Ex42}
      \end{minipage}}
    \caption{A simple proof by computation}
    \label{fig:Ex42E0}
  \end{figure}



%\afterpage{\clearpage}

\begin{figure}[th]
  \centering
  
  

\fbox{\begin{minipage}[h]{1.0\linewidth}
  Let us prove again the equality $\omega+42+\omega^2= \omega^2$. Let us recall that $\omega^2$ is an abbreviation of $\phi_0(2)$,
\emph{i.e} the third  additive principal ordinal.

\inputsnippets{Schutte/Ex42a}


Our proof is very different from the computational proof of
Figure~\ref{fig:Ex42E0}.
By definition of additive principal ordinals, 
it suffices to prove the inequality $\omega+42< \phi_0(2)$.

\inputsnippets{Schutte/Ex42b}

Since the set \textit{AP} of additive principals  is closed under addition
(by Lemma \textit{AP\_plus\_closed}), it suffices to prove the inequalities $\omega<\phi_0(2)$ and $42<\phi_0(2)$.

\inputsnippets{Schutte/Ex42d, Schutte/Ex42c, Schutte/Ex42e}

\end{minipage}
}
\caption{A proof interleaved with text (from the book)}
  \label{fig:Ex42-schutte}
\end{figure}

\begin{figure}
  \centering
  \inputsnippets{Pow/evalPow17LetIn}
  \caption{\label{fig:Pow-evalPow17LetIn}Automatically capturing the output of computations}
\end{figure}

\subsection{Continuous integration and deployment, opam packaging, etc.}
\section{Conclusion and perspectives}


\Hydras wants to be a bridge between scientific litterature (\emph{e.g.}~\cite{KP82, KS81, schutte} and proof assistant technology. For the mathematician, it may give a concrete view of the mathematical content, not only through full proofs, but also through significative computations: examples, functions associated with constructive proofs, etc. For the \coq user, it provides a consistent set of examples,
allowing to present and compare various formalization and proving techniques. It may be also used as a library, through opam packages.

It is also a medium sized library (more than 50K LOC), dependent on various tools and libraries of the \coq ecosystem. It may be also used to experiment new techniques of continuous maintenance of the code and its documentation.





%\TODO{General Conclusion}



%\subsection{Perspectives}

%\subsection{Further extensions}

We plan to extend our libraries in two main directions.

The \gaia project~\cite{Gaia}, also maintained on \community, contains a development by José Grimm~\cite{grimm:hal-00911710}. This library is dedicated  to the implementation in \coq (SSreflect) of books from  Bourbaki's Elements of Mathematics. It contains data structures which are compatible with our ordinal notations. 
Our plan is to build a  bridge between  the combinatorial results of \Hydras and the set-theoretic content of \gaia, and make possible the transfer of theorems between both libraries.

Another direction would be to write in \coq a formal proof of the original statement of Theorem~\ref{kp:thm2}, using O'Connor's formalization of Peano Arithmetic~\cite{Goedel}.

 \Hydras is not limited to the study of ordinal numbers and applications.
 For instance, we are also developing 
a module about efficient exponentiation algorithms, and hope
to extend our project to new topics.

%\section{Acknowledgments}
%\label{sect:acks}



\label{sect:bib}
\bibliographystyle{plain}
%\bibliographystyle{alpha}
%\bibliographystyle{unsrt}
%\bibliographystyle{abbrv}
\bibliography{../thebib}


\end{document}

