\usepackage{xspace, alltt}
\usepackage[utf8x]{inputenc}

\usepackage{fontspec}
%\setmainfont{DejaVu Serif}

\setmonofont{DejaVu Sans Mono}[Scale=MatchLowercase]

\newcommand{\TODO}[2][]{[\textcolor{red}{TODO (#1):} \emph{#2}]}
\newcommand{\coq}{Coq\xspace}
\newcommand{\coqdoc}{Coqdoc\xspace}
\newcommand{\coqmakefile}{\texttt{coq\_makefile}\xspace}
\newcommand{\community}{Coq-community\xspace}
\newcommand{\gaia}{Gaia\xspace}
\newcommand{\gaiahydra}{Gaia-Hydra\xspace}

\newcommand{\alectr}{Alectryon\xspace}
\newcommand{\equations}{Equations\xspace}
\newcommand{\Hydras}{Hydras \& Co$\text.$\xspace}
\newcommand{\make}{\texttt{make}\xspace}

\definecolor{orange}{rgb}{1.0,0.6,0.0}
\definecolor{darkorange}{rgb}{0.7,0.25,0.0}
\definecolor{plugincolor}{rgb}{0.6,0.0,0.6}
\definecolor{mathcolor}{rgb}{0.0,0.0,0.6}
\definecolor{coqstylecolor}{rgb}{0.0,0.0,01.0}
\definecolor{lightblue}{rgb}{0.2,0.2,1.0}
\definecolor{metavarcolor}{rgb}{0.5,0.0,1.0}
\definecolor{darkgreen}{rgb}{0.1,0.7,0.1}
\definecolor{answercolor}{rgb}{.08,.15,.8}
\definecolor{normalcolor}{rgb}{0.0,0.0,0.0}
\definecolor{exbluecolor}{rgb}{0.1,0.1,0.9}
\definecolor{dontlookcolor}{rgb}{0.5,0.5,0.5}
\definecolor{termcolor}{rgb}{0.0,0.1,0.9}
\definecolor{lookcolor}{rgb}{0.5,0.1,0.0}
\definecolor{prooftermcolor}{rgb}{0.3,0.1,1.0}
\definecolor{typecolor}{rgb}{1.0,0.6,0.0}
\definecolor{taccolor}{rgb}{0.70,0.10,0.0}
\definecolor{pink}{rgb}{0.8,0.6,0.6}
\definecolor{darkmagenta}{rgb}{0.4,0.0,0.6}
\definecolor{darkblue}{rgb}{0.0,0.0,0.6}
\definecolor{cyan}{rgb}{0.0,0.4,0.8}

\usepackage{tikz}
\usepackage{tikzsymbols}
\usepackage{pifont}
\usetikzlibrary{arrows}
\usetikzlibrary{shapes.geometric}
\usepackage{ifpdf}
\ifpdf
\usepackage{graphicx}
\else
\usepackage[dvips]{graphicx}
\fi

%%%% For Alectryon

\usepackage{texments}
\usepackage{./alectryon}
\usepackage{../pygments}

% Prevent breaks in the middle of syntactic units
\let\OldPY\PY
\def\PY#1#2{\OldPY{#1}{\mbox{#2}}}


%%% One hypothesis per line 
\makeatletter
\renewcommand{\alectryon@hyps@sep}{\alectryon@nl}
\makeatother

%%% \snippets{A,B,C,…} inputs a series of snippets as one block (with \itemsep
%%% between them).  A, B, C should be paths to files in snippets/.

\usepackage{etoolbox}
\makeatletter

\newcommand{\pathtomovies}{.}

\newcommand{\inputsnippets}[1]
{{\setlength{\itemsep}{1pt}\setlength{\parsep}{0pt}% Adjust spacing
    \alectryon@copymacros\begin{io}
      \forcsvlist{\item\@inputsnippet}{#1}
    \end{io}}}
\let\input@old\input % Save definition of \input
\newcommand{\@inputsnippet}[1]
{{\renewenvironment{alectryon}{}{}% Skip \begin{alectryon} included in snippet
    \input@old{{\pathtomovies}/#1}}}
\makeatother

% End of Alectryon stuff
%%%%%%%%%%%%%%%%%%%%%%%%%%%%%%%%%%%% 

%%%%%%% Specific macros

\newcommand{\canonseq}[2]{\mbox{$\{#1\}(#2)$}}
\newcommand{\rounds}{\mbox{\,\texttt{-+->}\,}}
\newcommand{\round}{\mbox{\,\texttt{-1->}\,}}
\newcommand{\rplus}[1]{\mbox{$\,\underset{#1}{\xrightarrow{\textcolor{white}{\,#1\,}}}\,$}}
