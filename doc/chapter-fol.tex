\chapter{First Order Logic (in construction)}
\label{chap:fol}

\section{Data types}

\subsection{Languages}

A \emph{language} is a structure composed of relation and function symbols, each symbol is given an \emph{arity} (number of arguments).

From~\href{../theories/html/hydras.Ackermann.fol.html}{Ackermann.fol}
\inputsnippets{fol/LanguageDef} 

In ~\href{../theories/html/hydras.Ackermann.Languages.html}{Ackermann.Languages}, we can find two examples of languages built with the usual symbols of arithmetic.

The first language, \texttt{LNT} (\emph{Language of Number Theory}) has just function symbols for $+$, $\times$, $0$ and successor.




The second language, LNN, is the language of NN (\emph{Natural Numbers}) and has
the same function symbols as LNT plus one relation symbol for less than, LT.

First, we declare two ranked alphabets. 
\inputsnippets{Languages/LNTDef1} 

In a second time, we build \texttt{LNT} and \texttt{LNN} by filling \texttt{Language}'s \texttt{arity} field.


\inputsnippets{Languages/LNTDef2} 


Let us show a few examples (from ~\href{../theories/html/hydras.MoreAck.fol_Examples.html}{MoreAck.fol\_Examples}).

\inputsnippets{fol_Examples/arityTest} 


\subsection{Terms}

Given a language $L$, we define the set of \emph{terms} and
$n$-\emph{tuples} of terms.

\inputsnippets{fol/TermDef} 


For instance the term $v_1+0$, where $v_1$ is a variable,
is represented by a term of type (\texttt{Term LNN}).

\inputsnippets{fol_Examples/v1Plus0} 

\subsection{Formulas}



The type of first order formulas over $L$ is defined 
in~\href{../theories/html/hydras.Ackermann.fol.html}{Ackermann.fol} as an inductive data type.

\inputsnippets{fol/FormulaDef}

The full abstract syntax of first-order formulas is completed 
with a few more notations. Note that we are studying \emph{classical} first-order logic, so the existential quantifier is derived from the universal one through negation. 

\inputsnippets{fol/FolFull, fol/folPlus}

\subsection{Examples}

In order to get more readable terms and formulas, we can define a few notations in ~\href{../theories/html/hydras.MoreAck.fol_Examples.html}{MoreAck.fol\_Examples}


\inputsnippets{fol_Examples/instantiations}

For instance, the term $v_1+0$ is written as below.

\inputsnippets{fol_Examples/v1Plus01}

Let us give two examples of first-order formulas.

\begin{itemize}
\item $\forall\,v_0\;v_1,\, v_0<v_1\, \longrightarrow\, v_0<v_1 + v_0$
\inputsnippets{fol_Examples/f1Example}
\item $\forall\,v_0,\,\exists\,v_1,\,v_0<v_1$
\inputsnippets{fol_Examples/f2Example}

\end{itemize}





\subsection{Decidability of equality}

Under the assumption that equality is decidable on 
function and relation symbols of any language $L$,
we get decidability of equality on terms and formulas.
Because of dependent types, the proofs are quite long and technical. The reader may consult them in \href{../theories/html/hydras.Ackermann.fol.html}{Ackermann.fol}

\begin{todo}
  Should we talk about the decomposition lemmas
\texttt{nilTerms} and \texttt{consTerms} ?
\end{todo}

\inputsnippets{fol/formDec1, fol/formDec2,
fol/formDec3,fol/formDec4,fol/formDec5}

\subsection{Induction principles on formula depth}

\subsubsection{Depth of a formula}

\inputsnippets{fol/depthDef}

\inputsnippets{fol_Examples/ltdepth1}

\begin{todo}
Motivate the induction principles based on depth. Compatible with term substitution and universal quantifier elimination.
\end{todo}




\begin{todo}
 Look for the principles which are really used in Ackermann or/and Goedel libraries, and comment them.
 Maybe skip the helpers (unused in other files)
\end{todo}