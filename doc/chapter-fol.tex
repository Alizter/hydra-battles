\chapter{First Order Logic (in construction)}
\label{chap:fol}

\section{Data types}

\subsection{Languages}

A \emph{language} is a structure composed of relation and function symbols, each symbol is given an \emph{arity} (number of arguments).

From~\href{../theories/html/hydras.Ackermann.fol.html}{Ackermann.fol}
\inputsnippets{fol/LanguageDef} 

In ~\href{../theories/html/hydras.Ackermann.Languages.html}{Ackermann.Languages}, we can find two examples of languages built with the usual symbols of arithmetic.

The first language, \texttt{LNT} (\emph{Language of Number Theory}) has just function symbols for $+$, $\times$, $0$ and successor.




The second language, LNN, is the language of NN (\emph{Natural Numbers}) and has
the same function symbols as LNT plus one relation symbol for less than, LT.

First, we declare two ranked alphabets. 
\inputsnippets{Languages/LNTDef1} 

In a second time, we build \texttt{LNT} and \texttt{LNN} by filling \texttt{Language}'s \texttt{arity} field.


\inputsnippets{Languages/LNTDef2} 


Let us show a few examples (from ~\href{../theories/html/hydras.MoreAck.fol_Examples.html}{MoreAck.fol\_Examples}).

\inputsnippets{fol_Examples/arityTest} 


\subsection{Terms}

Given a language $L$, we define the set of \emph{terms} and
$n$-\emph{tuples} of terms.

\inputsnippets{fol/TermDef} 

For instance the term $v_1+0$, where $v_1$ is a variable,
is represented by a term of type (\texttt{Term LNN}).

\inputsnippets{fol_Examples/v1Plus0} 




