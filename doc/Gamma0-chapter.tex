\chapter{The Ordinal \texorpdfstring{$\Gamma_0$}{Gamma0} (first draft)}


\emph{This chapter and the files it presents are still very incomplete, considering the impressive properties of $\Gamma_0$~\cite{Gallier91}.  We hope to add new material soon, and accept contributions!}

\begin{todo}
  Build a T2Bridge! (note: Gamma0's wellfoundedness is missing in \gaia (?).
\end{todo}

\section{Introduction}
We present a notation system for the ordinal $\Gamma_0$, following Chapter V, Section 14 of~\cite{schutte}: ``A notation system for the ordinals $<\Gamma_0$''.
We try to be as close as possible to Schütte's text and usual practices of \coq{} developments.

The ordinal $\Gamma_0$ is defined in Section 13 of ~\cite{schutte} as the least \emph{strongly critical ordinal}. It is widely known as the \emph{Feferman-Schütte ordinal}.


Section V, 13 of~\cite{schutte} defines \emph{strongly critical} and
\emph{maximal $\alpha$-critical} ordinals: 

\begin{itemize}
\item $\alpha$ is strongly critical if
$\alpha$ is $\alpha$-critical,
\item $\gamma$ is maximal $\alpha$-critical if $\gamma$ is $\alpha$-critical, and, for all $\xi>\alpha$, $\gamma$ is not $\xi$-critical.

\end{itemize}





\vspace{4pt}

\noindent\emph{From \href{../theories/html/hydras.Schutte.Critical.html\#strongly_critical}%
{\texttt{Schutte.Critical}}}

\input{movies/snippets/Critical/Gamma0Def}


\index{hydras}{Projects}
\begin{project}
Prove that a (countable)  ordinal $\alpha$ is strongly critical iff 
$\phi_\alpha(0)=\alpha$ (Theorem 13.13 of~\cite{schutte} ). 
\end{project}


\index{hydras}{Projects}
\begin{project}
Prove that the set of strongly critical ordinals is unbounded and closed (Theorem 13.14 of~\cite{schutte} ). Thus this set is not empty,  hence has a least element. Otherwise, the definition of $\Gamma_0$ above would be useless.
\end{project}




In the present version of this development, we  only study $\Gamma_0$ as a notation system, much more powerful than the ordinal notation for $\epsilon_0$.

%\index{Projects}
%
%\begin{project}
% Schûtte's section 13 of~\cite{schutte} contain several definitions and lemmas which give another view on $\Gamma_0$.  We leave it as a project to implement them in \coq{}.  
%\end{project}




\section{The type \texttt{T2} of ordinal terms}

The notation system for ordinals less than $\gamma_0$ comes from the following theorem of~\cite{schutte}, where $\psi\,\alpha$ is the ordering function 
of the set of maximal $\alpha$-critical ordinals.

\index{hydras}{Library Gamma0!Types!T2}

\begin{quote}
  Any ordinal $\not= 0$ which is not strongly critical can be expressed in terms of $+$ and $\psi$.
\end{quote}

\index{hydras}{Projects}
\begin{project}
This theorem is not formally proved in this development yet. It should be!
\end{project}


Like in Chapter~\ref{chap:T1}, we define an inductive type with two constructors, one for $0$, the other for the construction $\psi(\alpha,\beta)\times(n+1)+\gamma$, adapting a Manolios-Vroon-like notation~\cite{Manolios2005} to
\emph{Veblen normal forms}.
\label{types:T2}

\noindent\emph{From \href{../theories/html/hydras.Gamma0.T2.html\#T2}%
{\texttt{Gamma0.T2}}}

\inputsnippets{T2/T2Def, T2/psiDef}



Like in chapter~\ref{chap:T1}, we get familiar with the type \texttt{T2} by recognizing simple constructs like finite ordinals, $\omega$, etc., as inhabitants of \texttt{T2}.

\input{movies/snippets/T2/Notations}

\section{A strict order on T2}

Let us define a strict order on type \texttt{T2}. The following definition is 
an adaptation of Schütte's, taking into account the multiplications by a natural number (inspired by~\cite{Manolios2005}, and also present in \texttt{T1}).

\label{sect:t2-lt-def}

\input{movies/snippets/T2/ltDef}


Seven constructors! In order to get accustomed with this definition, let us look at a small set of examples, covering all the constructors of \texttt{lt}.


\input{movies/snippets/T2/ltExamples}


\index{hydras}{Projects}
\begin{project}
Write a tactic that solves automatically goals of the form (\texttt{$\alpha$ t2< $\beta$}), where $\alpha$ and $\beta$ are closed terms of type \texttt{T2}.
\end{project}

\section{Veblen normal form}
\begin{definition}
  A term of the form $\psi(\alpha_1,\beta_1)\times n_1+ \psi(\alpha_2,\beta_2)\times n_2+\dots+\psi(\alpha_k,\beta_k)\times n_k$ is said to be in
 \emph{[Veblen] normal form} if for every $i<n$, $\psi(\alpha_i,\beta_i)<\psi(\alpha_{i+1},\beta_{i+1})$, all the $\alpha_i$ and $\beta_i$ are in normal form, and all the $n_i$ are strictly positive integers.
\end{definition}

\input{movies/snippets/T2/nfDef}



Let us look at some positive examples (we have to prove some inversion lemmas before proving counter-examples).

\input{movies/snippets/T2/nfExamples}


\subsection{Length of a term}

The notion of \emph{term length} is introduced by Schütte as a helper for proving (at least) the \emph{trichotomy} property and transitivity of the strict order \texttt{lt} on \texttt{T2}. These properties are proved by induction on length.


\subsection{Trichotomy}

\emph{Trichotomy} is another name for the well-known property of decidable total ordering (like Standard Library's \texttt{Compare\_dec.lt\_eq\_lt\_dec}).

We first prove by induction on $l$ the following lemma:

\vspace{4pt}

\noindent\emph{From \href{../theories/html/hydras.Gamma0.Gamma0\#tricho_aux}%
{\texttt{Gamma0.Gamma0}}}

\input{movies/snippets/Gamma0/tricho}

\input{movies/snippets/Gamma0/compare}


With the help of \texttt{compare}, we get a boolean version of \texttt{nf}
(being in Veblen normal form).

\input{movies/snippets/Gamma0/nfb}





\section{Main functions on \texttt{T2}}

\subsection{Successor}
The successor function is defined by structural recursion.

\noindent\emph{From \href{../theories/html/hydras.Gamma0.T2.html\#succ}%
{\texttt{Gamma0.T2}}}

\input{movies/snippets/T2/succDef}


\subsection{Addition}

Like for Cantor normal forms (see Sect.~\ref{sect:infix-plus-T1}),  the definition of addition in \texttt{T2}  requires comparison between ordinal terms.

\noindent\emph{From \href{../theories/html/hydras.Gamma0.Gamma0T2.html\#succ}%
  {\texttt{Gamma0.Gamma0}}}

\input{movies/snippets/Gamma0/plusDef}



\subsection{The Veblen function \texorpdfstring{$\phi$}{\texttt{phi}}}

The enumeration function of critical ordinals, presented in Sect.~\vref{sect:phi-schutte}, is recursively defined in type \texttt{T2}.

\input{movies/snippets/Gamma0/phiDef}

Despite its complexity, the function \texttt{phi} is well adapted to proofs by simplification or computation.


The relation between the constructor $\psi$ and the function $\phi$ is
studied in~\cite{schutte}, and partially implemented in this development.
\emph{Please contribute!}
 
For instance, the following theorem states that, if $\gamma$ is the sum of a limit ordinal $\beta$ and a finite ordinal $n$, and $\beta$ is a fixpoint of
$\phi(\alpha)$, then $\psi(\alpha,\gamma)=\phi_\alpha(\gamma+1)$.

\input{movies/snippets/Gamma0/phiPsi}
\input{movies/snippets/Gamma0/Ex9}




On the other hand, $\phi$ can be expressed in terms of $\psi$.

\input{movies/snippets/Gamma0/phiOfPsi}
\input{movies/snippets/Gamma0/Ex10}

\index{hydras}{Projects}
\begin{project}
Please study a way to pretty print ordinal terms in Veblen normal form (see Section~\vref{sect:ppT1}).
\end{project}

\section{An ordinal notation for \texorpdfstring{$\Gamma_0$}{\texttt{Gamma0}}}

In order to consider type \texttt{T2} as an ordinal notation, we have to build an instance of class \texttt{ON} (See Definition page~\pageref{types:ON}).

First, we define a type that contains only terms in Veblen normal form, and redefine \texttt{lt} and \texttt{compare} by delegation (see for comparison the construction of type \texttt{E0} in Sect.~\vref{sect:E0-def}).

\input{movies/snippets/Gamma0/G0a}
\input{movies/snippets/Gamma0/G0b}


Then, we buils an instance of class \texttt{ON}. 
function \texttt{compare} is correct.

\input{movies/snippets/Gamma0/ltSto}
\input{movies/snippets/Gamma0/ltWf}
\input{movies/snippets/Gamma0/ONGamma0}
\input{movies/snippets/Gamma0/G0z}


\begin{remark}
The proof of \texttt{lt\_wf} has been written by \'Evelyne Contejean, using her library on the recursive path ordering (see also remark~\vref{remark:a3pat}).
\end{remark}

\index{hydras}{Projects}
\begin{project}
Prove that \texttt{Epsilon0} (page~\pageref{instance-epsilon0})
is a sub-notation system of \texttt{Gamma0}.

Prove that the implementations of \texttt{succ}, \texttt{+}, $\phi_0$, etc.
are compatible in both notation systems.

Note that a function \texttt{T1\_inj} from \texttt{T1} to \texttt{T2} has already been defined. It may help to complete the task.



\noindent\emph{From \href{../theories/html/hydras.Gamma0.T2.html\#T1_to_T2}%
{\texttt{Gamma0.T2}}}

\input{movies/snippets/T2/T1ToT2}

\end{project}

\begin{project}
Prove that the notation system \texttt{Gamma0} is a correct implementation 
of the segment $[0,\Gamma_0)$ of the set of countable ordinals.
\end{project}
