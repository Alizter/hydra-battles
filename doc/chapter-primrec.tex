\chapter{Primitive Recursive Functions}

\section{Introduction}
The definition of primitive recursive functions we use come
from Russel O'Connors formalization in \coq{} of
G\"odel's incompleteness theorems~\cite{OConnor05}, now hosted in
the \texttt{theories/ordinals/Ackermann} directory.

This chapter contains some comments on Russel's library, as well as a few extensions.

\section{First look at the Ackermann library}

O'Connor's library on Gödel's incompleteness theorems contains a little more 
than 45K lines of scripts. The part dedicated to primitive recursive functions and Peano arithmetics is 32K lines long and is originally structured in 38 modules.

Thus, we propose a slow exploration of this library, through examples and exercises.

\section{Basic Libraries}

The formal definition of primitive recursive functions lies in the library
\href{../theories/html/hydras.Ackermann.primRec.html}{Ackermann.primRec},
with preliminary definitions in 
\href{../theories/html/hydras.Ackermann.extEqualNat.html}{Ackermann.extEqualNat}
and
\href{../theories/html/hydras.Ackermann.misc.html}{Ackermann.misc}.

\subsection{Functions of arbitrary arity}

The library allows us to consider primitive functions on \texttt{nat}, with any number of arguments, in 
curried form. This is made possible in 
\href{../theories/html/hydras.Ackermann.extEqualNat.html}{Ackermann.extEqualNat} by the following definition:

\begin{Coqsrc}
Fixpoint naryFunc (n : nat) : Set :=
  match n with
  | O => nat
  | S n => nat -> naryFunc n
  end.
\end{Coqsrc}

For instance (\texttt{naryFunc 1}) is convertible to \texttt{nat -> nat} and (\texttt{naryFunc 3})
to \texttt{nat -> nat -> nat -> nat}.

\vspace{4pt}
\noindent
\emph{From\href{../theories/html/hydras.Ackermann.PrimRecExamples.html}{Ackermann.PrimRecExamples}}.
\begin{Coqsrc}
Require Import primRec.
Import extEqualNat.

Compute naryFunc 3.
\end{Coqsrc}

\begin{Coqanswer}
= nat -> nat -> nat -> nat
  : Set  
\end{Coqanswer}

Thus, the function \texttt{plus} has also type (\texttt{naryFunc 2}).

Likewise, arbitrary boolean predicates may have an arbitrary number of arguments. The dependent type
(\texttt{naryRel $n$}), defined in the same way as \texttt{naryFunc}, is the type of $n$-ary functions from
\texttt{nat} into \texttt{bool}.

\begin{Coqsrc}
Compute naryRel 2.
\end{Coqsrc}

\begin{Coqanswer}
 = nat -> nat -> bool
     : Set
\end{Coqanswer}

The magic of dependent types makes it possible to define recursively extensional equality between functions of the same arity.

\index{Coq!Dependent types}
\index{Coq!Dependently typed functions}
\vspace{4pt}
\noindent
\emph{From \href{../theories/html/hydras.Ackermann.extEqualNat.html}{Ackermann.extEqualNat}}
\begin{Coqsrc}
Fixpoint  extEqual (n : nat) : forall  (a b : naryFunc n), Prop :=
  match n with
    0 => fun a b => a = b
  | S p => fun a b => forall c, extEqual p (a c) (b c)
  end.
\end{Coqsrc}

\begin{Coqsrc}
Compute extEqual 2.
\end{Coqsrc}

\begin{Coqanswer}
     = fun a b : naryFunc 2 => forall x x0 : nat, a x x0 = b x x0
     : naryFunc 2 -> naryFunc 2 -> Prop
 \end{Coqanswer}
 
\begin{Coqsrc}
Example extEqual_ex1 : extEqual 2 mult (fun x y =>  y * x + x -x) .
Proof.
  intros x y.
\end{Coqsrc}

\begin{Coqanswer}
  x, y : nat
  ============================
  extEqual 0 (x * y) (y * x)
\end{Coqanswer}

\begin{Coqsrc}
  cbn.
\end{Coqsrc}

\begin{Coqanswer}
1 subgoal (ID 10)
  
  x, y : nat
  ============================
  x * y = y * x + x - x
\end{Coqanswer}

\begin{Coqsrc}
  rewrite <- Nat.add_sub_assoc, Nat.sub_diag.
  - ring.
  - apply le_n.  
Qed.
\end{Coqsrc}

\subsection{A Data-type for Primitive Recursive Functions}

The traditional definition of the set of primitive recursive functions is structured as an inductive definition 
in five rules: three base cases, and two construction rules. Primitive recursive functions are total functions from $\mathbb{N}^n$ to
$\mathbb{N}$, for some $n\in\mathbb{N}$.

\begin{description}
  \item[zero] the constant function of value $0$ is primitive recursive.
\item[S] The successor function $S:\mathbb{N}\rightarrow\mathbb{N}$ is primitive recursive.
 \item[projections] For any pair $0\leq i<n$, the projection $\pi_{n,i}: \mathbb{N}^n\rightarrow\mathbb{N}$, defined by $\pi_{n,i}(x_0,x_1,\dots,x_{n-1})=x_i$, is primitive recursive.
\item[composition] For any $n$ and $m$, if $h: \mathbb{N}^m\rightarrow\mathbb{N}$, and
$g_0,\dots, g_{m-1}$ are primitive recursive of $n$ arguments, then the function which maps any
tuple $(x_0,\dots,x_{n-1})$ to $h(g_0(x0,\dots,x_{n-1}),\dots, g_{m-1}(x0,\dots,x_{n-1}))$ is primitive recursive.
\item[primitive recursion]
If $g: \mathbb{N}^n\rightarrow\mathbb{N}$ and $h: \mathbb{N}^{n+2}\rightarrow\mathbb{N}$ are primitive recursive, then the function from $\mathbb{N}^{n+1}$ into $\mathbb{N}$ defined by
\begin{align}
f(0,x_1,\dots,x_n)&=g(x_1,\dots,x_n)\\
f(S(p),x_1,\dots,x_n)&=h(p,f(p, x_1,\dots,x_n),  x_1,\dots,x_n)
\end{align} 
is primitive recursive.
\end{description}

O'Connor's formalization of primitive recursive functions takes the form of two mutually inductive dependent data types, each constructor of which is associated with one of these  rules.
These two types are (\texttt{PrimRec $n$}) (primitive recursive functions of $n$ arguments), and
(\texttt{PrimRecs $n$ $m$}) ($m$-tuples of primitive recursive functions of $n$ arguments).

\index{Coq!Dependent types}
\index{Coq!Mutually inductive types}

\vspace{4pt}
\noindent
\emph{From \href{../theories/html/hydras.Ackermann.primRec.html}{Ackermann.primRec}}
\begin{Coqsrc}
Inductive PrimRec : nat -> Set :=
  | succFunc : PrimRec 1
  | zeroFunc : PrimRec 0
  | projFunc : forall n m : nat, m < n -> PrimRec n
  | composeFunc :
      forall (n m : nat) (g : PrimRecs n m) (h : PrimRec m), PrimRec n
  | primRecFunc :
      forall (n : nat) (g : PrimRec n) (h : PrimRec (S (S n))), 
      PrimRec (S n)
with PrimRecs : nat -> nat -> Set :=
  | PRnil : forall n : nat, PrimRecs n 0
  | PRcons : forall n m : nat, PrimRec n -> PrimRecs n m -> 
                   PrimRecs n (S m).
\end{Coqsrc}

\subsection{A little bit of Semantics} 
Please note that inhabitants of type (\texttt{PrimRec $n$}) are not \coq{} functions like \texttt{Nat.mul}, or factorial, etc. The data-type (\texttt{PrimRec $n$}) is indeed an abstract syntax for the language of primitive recursive functions. The bridge between this language and the word of usual functions
is an interpretation function (\texttt{evalprimRec $n$})  from  (\texttt{PrimRec $n$}) to  (\texttt{naryFunc $n$}) (with the helper (\texttt{evalprimRecS $n$ $m$})
from  (\texttt{PrimRecs $n$ $m$}) to  (\texttt{Vector.t (naryFunc $n$) $m$}).

Both functions are mutually defined through dependent pattern matching. We advise the readers who 
would feel uneasy with dependent types to consult Adam Chlipala's \emph{cpdt}  book~\cite{chlipalacpdt2011}. We leave it to the reader  to look at the helper functions in
\href{../theories/html/hydras.Ackermann.primRec.html}{Ackermann.primRec}.


\begin{Coqsrc}
Fixpoint evalPrimRec (n : nat) (f : PrimRec n) {struct f} : 
 naryFunc n :=
  match f in (PrimRec n) return (naryFunc n) with
  | succFunc => S
  | zeroFunc => 0
  | projFunc n m pf => evalProjFunc n m pf
  | composeFunc n m l f =>
      evalComposeFunc n m (evalPrimRecs _ _ l) (evalPrimRec _ f)
  | primRecFunc n g h =>
      evalPrimRecFunc n (evalPrimRec _ g) (evalPrimRec _ h)
  end
 
 with evalPrimRecs (n m : nat) (fs : PrimRecs n m) {struct fs} :
 Vector.t (naryFunc n) m :=
  match fs in (PrimRecs n m) return (Vector.t (naryFunc n) m) with
  | PRnil a => Vector.nil  (naryFunc a)
  | PRcons a b g gs =>
       Vector.cons _ (evalPrimRec _ g) _  (evalPrimRecs _ _ gs)
  end.
\end{Coqsrc}
Let us take as example the following term\footnote{Of course, we never typed this term \emph{verbatim}, but obtained as the result of some computation we leave it the reader to guess and reproduce.}:

\begin{Coqsrc}
Example weirdPR : PrimRec 1 :=
primRecFunc 0
  (composeFunc 0 1 (PRcons 0 0 zeroFunc (PRnil 0)) succFunc)
  (composeFunc 2 2
    (PRcons 2 1
      (composeFunc 2 1
         (PRcons 2 0 (projFunc 2 1 (le_n 2))
                 (PRnil 2))
         succFunc)
      (PRcons 2 0
        (composeFunc 2 1
          (PRcons 2 0
             (projFunc 2 0
                       (le_S 1 1 (le_n 1)))
             (PRnil 2))
          (projFunc 1 0 (le_n 1))) (PRnil 2)))
    (primRecFunc 1 (composeFunc 1 0 (PRnil 1) zeroFunc)
       (composeFunc 3 2
         (PRcons 3 1
            (projFunc 3 1 (le_S 2 2 (le_n 2)))
            (PRcons 3 0 (projFunc 3 0
                          (le_S 1 2
                                (le_S 1 1 (le_n 1))))
                    (PRnil 3)))
         (primRecFunc 1 (projFunc 1 0 (le_n 1))
                      (composeFunc 3 1
                          (PRcons 3 0
                                  (projFunc 3 1 (le_S 2 2 (le_n 2)))
                                  (PRnil 3))
                          succFunc))))). 
\end{Coqsrc}

Let us now interpret this term as an arithmetic function.

\begin{Coqsrc}
Example  mystery_fun : nat _> nat := evalPrimRec 1 weirdPR.

Compute List.map mystery_fun (0::1::2::3::4::5::6::nil) : list nat.
\end{Coqsrc}

\begin{Coqanswer}
 = 1 :: 1 :: 2 :: 6 :: 24 :: 120 :: 720 :: nil
     : list nat
\end{Coqanswer}

Ok, the term \texttt{weirdPR} looks to be a primitive recursive definition of the factorial function, although we haven't proved this fact yet. Fortunately, we will see in the following sections much simpler ways to prove that a given function is primitive recursive, whithout looking at an unreadable term.
