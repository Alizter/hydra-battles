\chapter{Primitive recursive functions}

\section{Introduction}
Primitive recursive functions are a small class of total functions, corresponding to the expressive power of a simple imperative programming language without \textbf{while} loops, in which every program execution terminates.

Primitive recursive functions are total functions from $\mathbb{N}^n$ to
$\mathbb{N}$, for some $n\in\mathbb{N}$. Note that not all 
total $n$-ary recursive functions are primitive recursive
(see for instance Sect.~\vref{sect:ack-not-PR}).

The traditional definition of the set of primitive recursive functions is structured as an inductive definition 
in five rules: three base cases, and two recursive construction rules. 

\begin{description}
  \item[zero] the constant function of value $0$ is primitive recursive.
\item[S] The successor function $S:\mathbb{N}\rightarrow\mathbb{N}$ is primitive recursive.
 \item[projections] For any pair $0< i\leq n$, the projection $\pi_{i,n}: \mathbb{N}^n\rightarrow\mathbb{N}$, defined by $\pi_{i,n}(x_1,x_2,\dots,x_{n})=x_i$, is primitive recursive.
\item[composition] For any $n$ and $m$, if $h: \mathbb{N}^m\rightarrow\mathbb{N}$, and
$g_0,\dots, g_{m-1}: \mathbb{N}^n\rightarrow\mathbb{N}$ are primitive recursive of $n$ arguments, then the function which maps any
tuple $(x_0,\dots,x_{n-1})$ to $h(g_0(x0,\dots,x_{n-1}),\dots, g_{m-1}(x0,\dots,x_{n-1})): \mathbb{N}^n\rightarrow\mathbb{N}$ is primitive recursive.
\item[primitive recursion]
If $g: \mathbb{N}^n\rightarrow\mathbb{N}$ and $h: \mathbb{N}^{n+2}\rightarrow\mathbb{N}$ are primitive recursive, then the function from $\mathbb{N}^{n+1}$ into $\mathbb{N}$ defined by
\begin{align}
f(0,x_1,\dots,x_n)&=g(x_1,\dots,x_n)\\
f(S(p),x_1,\dots,x_n)&=h(p,f(p, x_1,\dots,x_n),  x_1,\dots,x_n)
\end{align} 
is primitive recursive.
\end{description}


Please note the use of dots: $\ldots$ in the definition above. 
Dots are not part of \gallina's syntax. Thus, the formal definition of the set of primitive recursive function will have to overcome this representation problem.

  We present in this chapter a formalization of  primitive recursive functions, taken from  Russel O'Connor's formalization in \coq{} of
G\"odel's incompleteness theorems~\cite{OConnor05}.

\begin{remark}
 The theory of primitive recursive function is now hosted in
the \texttt{theories/ordinals/Ackermann} directory.
The specific part on G\"odel's theorem,  is also on
coq-community (\url{https://github.com/coq-community/goedel}) and requires the 
\href{https://github.com/coq-community/pocklington}{Pocklington library} for lemmas on primality.
\end{remark}

This chapter contains some comments on Russel's library, as well as a few extensions.
Contributions (under the form of comments, new examples or exercises) are welcome!. 



\section{First look at the Ackermann library}

O'Connor's library on Gödel's incompleteness theorems contains a little more 
than 45K lines of scripts. The part dedicated to primitive recursive functions and Peano arithmetics is 32K lines long and is originally structured in 38 modules.
Thus, we propose a partial exploration of this library, through examples and exercises. Our additions to the original library --- mainly examples and counter-examples ---,
are stored in the directory \texttt{theories/ordinals/MoreAck}.

In particular, the library \href{../theories/html/hydras.MoreAck.AckNotPR.html}{MoreAck.AckNotPR} contains the well-known  proof that the Ackermann function is not primitive recursive (see Section~\vref{sect:ack-not-PR}).
MoreOver, the library \href{../theories/html/hydras.Hydra.Hydra_Theorems.html}{Hydra.Hydra\_Theorems} contains 
a proof that the length of an hydra battle (according to the initial replication factor) is not primitive recursive in general.

\section{Basic definitions}
\index{maths}{Primitive recursive functions}

The formal definition of primitive recursive functions lies in the library
\href{../theories/html/hydras.Ackermann.primRec.html}{Ackermann.primRec},
with preliminary definitions in 
\href{../theories/html/hydras.Ackermann.extEqualNat.html}{Ackermann.extEqualNat}
and
\href{../theories/html/hydras.Ackermann.misc.html}{Ackermann.misc}.

\subsection{Functions of arbitrary arity}

The  \texttt{primRec} library allows us to consider primitive functions on \texttt{nat}, with any number of arguments, in 
curried form. This is made possible in 
\href{../theories/html/hydras.Ackermann.extEqualNat.html}{Ackermann.extEqualNat} by the following definition:

\index{primrec}{Types!naryFunc}
\input{movies/snippets/extEqualNat/naryFunc.tex}

For instance (\texttt{naryFunc 1}) is convertible to \texttt{nat -> nat} and (\texttt{naryFunc 3})
to \texttt{nat -> nat -> nat -> nat}.

\vspace{4pt}
\noindent
\emph{From \href{../theories/html/hydras.MoreAck.PrimRecExamples.html}{MoreAck.PrimRecExamples}}.

\input{movies/snippets/PrimRecExamples/naryFunc3}
\input{movies/snippets/PrimRecExamples/checknaryFunc}


Likewise, arbitrary boolean predicates may have an arbitrary number of arguments. The dependent type
(\texttt{naryRel $n$}), defined in the same way as \texttt{naryFunc}, is the type of $n$-ary functions from
\texttt{nat} into \texttt{bool}.

\input{movies/snippets/PrimRecExamples/naryRel2}

The magic of dependent types makes it possible to define recursively extensional equality between functions of the same arity.

\index{coq}{Dependent types}
\index{coq}{Dependently typed functions}
\vspace{4pt}
\noindent
\emph{From \href{../theories/html/hydras.Ackermann.extEqualNat.html}{Ackermann.extEqualNat}}

\index{primrec}{Predicates!extEqual}

\input{movies/snippets/extEqualNat/extEqualDef}


\input{movies/snippets/PrimRecExamples/extEqual2a}

  Getting rid of the term \texttt{x-x}, we generate two easy-to-solve sub-goals.

\vspace{6pt}
\noindent
\input{movies/snippets/PrimRecExamples/extEqual2b}
  
\subsection{A Data-type for Primitive Recursive Functions}

O'Connor's formalization of primitive recursive functions takes the form of two mutually inductive dependent data types, each constructor of which is associated with one of these  rules.
These two types are (\texttt{PrimRec $n$}) (primitive recursive functions of $n$ arguments), and
(\texttt{PrimRecs $n$ $m$}) ($m$-tuples of primitive recursive functions of $n$ arguments).


\index{coq}{Dependent types}
\index{coq}{Mutually inductive types}

\index{primrec}{Types!PrimRec}
\index{primrec}{Types!PrimRecs}
\label{def:Primrec}
\vspace{4pt}
\noindent
\emph{From \href{../theories/html/hydras.Ackermann.primRec.html}{Ackermann.primRec}.}

\input{movies/snippets/primRec/PrimRecDef}

\begin{remark}
\label{projFunc-order-of-args}
Beware of the conventions used in the \texttt{primRec} library!
The constructor (\texttt{projFunc $n$ $m$})  is associated with the projection $\pi_{n-m,n}$ and \emph{not}
$\pi_{n, m}$.
For instance, the projection $\pi_{2,5}$ defined by $\pi_{2,5}(a,b,c,d,e)=b$ corresponds to the term
(\texttt{projFunc 5 3 H}), where \texttt{H} is a proof of $3<5$.
 This fact is reported in the comments of \texttt{primRec.v}. We presume that this convention makes it easier to define the evaluation function (\texttt{evalProjFunc $n$}) (see the next sub-section). Trying the other convention is left as an exercise.
\end{remark}



\subsection{A little bit of semantics} 
Please note that inhabitants of type (\texttt{PrimRec $n$}) are not \coq{} functions like \texttt{Nat.mul}, or factorial, etc. The data-type (\texttt{PrimRec $n$}) is indeed an abstract syntax for the language of primitive recursive functions. The bridge between this language and the word of usual functions
is an interpretation function (\texttt{evalprimRec $n$})  of type
$\texttt{PrimRec}\,n \rightarrow  \texttt{naryFunc}\,n$.
This function is defined by mutual recursion,  together with the  function 
(\texttt{evalprimRecS $n$ $m$}) of type 
$\texttt{PrimRecs}\,n\,m \rightarrow  \texttt{Vector.t}\,(\texttt{naryFunc}\,n)\,m$.

\index{primrec}{Functions!evalPrimRec}
\index{primrec}{Functions!evalPrimRecs}

\index{coq}{Dependent pattern matching}
Both functions are mutually defined through dependent pattern matching. We advise the readers who 
would feel uneasy with dependent types to consult Adam Chlipala's \emph{cpdt}  book~\cite{chlipalacpdt2011}. We leave it to the reader  to look also at the helper functions in
\href{../theories/html/hydras.Ackermann.primRec.html}{Ackermann.primRec}.

\vspace{4pt}

\input{movies/snippets/primRec/evalPrimRecDef}

\vspace{4pt}

Looks complicated? The following examples show that, when
the arity is fixed, these definitions behave well w.r.t. 
\coq's reduction rules. Moreover, they make the interpretation functions more ``concrete''.

\vspace{4pt}
\noindent
\emph{From \href{../theories/html/hydras.MoreAck.PrimRecExamples.html}{MoreAck.PrimRecExamples}.}

\input{movies/snippets/PrimRecExamples/evalPrimRecEx}

\vspace{4pt}
\noindent

Another example?
Let us consider the following term\footnote{Of course, we never typed this term \emph{verbatim}; we obtained it by an interactive proof the reader will be able to make after 
reading Sect.\vref{sect:proofs-of-isPR}.}:

\label{sect:bigfac}

\input{movies/snippets/PrimRecExamples/bigPRa}


Let us now interpret this term as an arithmetic function.

\vspace{4pt}
\noindent

\input{movies/snippets/PrimRecExamples/bigPRb}


After this test, the term \texttt{bigPR} looks to be a primitive recursive definition of the factorial function, although we haven't proved this fact yet. Fortunately, we will see in the following sections simple ways to prove that a given function is primitive recursive, whithout building such an unreadable term.

\section{Proving that a given arithmetic function is primitive recursive}
\label{sect:proofs-of-isPR}

The example in the preceding section clearly shows that, in order to prove that a given arithmetic function
(defined in \gallina{} as usual) is primitive recursive, trying to give  by hand a term  of type (\texttt{PrimRec $n$}) is not a good method, since such terms may be huge and complex, even for simple arithmetic functions. The method proposed in Library \texttt{primRec} is the following one:

\begin{enumerate}
\item Define a type corresponding to the statement "the function \texttt{$f$:naryFunc $n$} is primitive recursive ''.
\item Prove handy lemmas which may help to prove that a given function is primitive recursive.
\end{enumerate}

Thus, the proof that a function, like \texttt{factorial}, is primitive recursive may be interactive, whithout having to type complex terms at any step of the development.

\subsection{The predicate \texttt{isPR}}

\index{primrec}{Predicates!isPR}
\index{coq}{Extensionnaly equal functions}

Let $f$ be an arithmetic function of arity $n$. We say that $f$ is primitive recursive if $f$ is \textbf{extensionnaly}
equal to the interpretation of some term of type \texttt{PrimRec $n$}. 

\vspace{4pt}
\noindent
\emph{From \href{../theories/html/hydras.Ackermann.primRec.html}{Ackermann.primRec}.}

\input{movies/snippets/primRec/isPRDef}

The library \texttt{primRec} contains a large catalogue of lemmas allowing to prove statements 
of the form (\texttt{isPR $n$ $f$}). We won't list all these lemmas here, but give a few examples of
how they may be applied.

\begin{remark}
In the library \texttt{primRec}, all these lemmas are opaque (registered with \texttt{Qed}. Thus they do not allow the user to look at the witness of a proof of a \texttt{isPR} statement. Our example of page\pageref{sect:bigfac} was built using a  copy of \texttt{primRec.v} where many \texttt{Qed}s have been replaced with
\texttt{Defined}s.

If it does not cause compatibility problems (with 
\href{https://github.com/coq-community/goedel}{goedel library} for instance), we plan to make all theses lemmas transparent.
\end{remark}

\subsubsection{Elementary proofs of \texttt{isPR} statements}

The constructors \texttt{zeroFunc}, \texttt{succFunc},  and \texttt{projFunc} of type
\texttt{PrimRec} allows us to write trivial proofs of primitive recursivity. 
Although  the following lemmas are already proven in 
\href{../theories/html/hydras.Ackermann.primRec.html}{Ackermann.primRec},
we wrote alternate proofs in 
\href{../theories/html/hydras.MoreAck.PrimRecExamples.html}%
{Ackermann.MoreAck.PrimRecExamples.v}, in order to illustrate the main proof patterns.

\input{movies/snippets/PrimRecExamples/zeroIsPR}


Likewise, we prove that the successor function on \texttt{nat} is primitive recursive too.

\input{movies/snippets/PrimRecExamples/SuccIsPR}



Projections are proved primitive recursive, case by case (many examples in 
\href{../theories/html/hydras.Ackermann.primRec.html}{Ackermann.primRec}).
\emph{Please notice again that the name of the projection follows the mathematical tradition, 
whilst the arguments of  \texttt{projFunc} use another convention (\emph{cf} remark~\vref{projFunc-order-of-args}).}


\input{movies/snippets/PrimRecExamples/pi25IsPR}



Please note that the projection $\pi_{1,1}$ is just the identity on \texttt{nat}, and is realized by 
(\texttt{projFunc 1 0}).


\vspace{4pt}
\noindent
\emph{From \href{../theories/html/hydras.Ackermann.primRec.html}{Ackermann.primRec}.}

\input{movies/snippets/primRec/idIsPR}

\subsubsection{Using function composition}

Let us look at the proof that any constant $n$ of type \texttt{nat} has type (\texttt{PR 0})
(lemma  \texttt{const1\_NIsPR} of \texttt{primRec}). We carry out a proof by induction on $n$, the base case of which is already proven.
Now, let us assume $n$ is \texttt{PR $n$}, with $x:\texttt{PrimRec}\,0$ as a ``realizer''.
Thus we would like to compose this constant function with the unary successor function.

This is exactly the role of the instance \texttt{composeFunc 0 1} of the dependently typed
function \texttt{composeFunc}, as shown by the following lemma.

\vspace{4pt}
\input{movies/snippets/PrimRecExamples/compose01}


\vspace{4pt}
Thus, we get a quite simple proof of \texttt{const1\_NIsPR}.


\vspace{4pt}
\noindent
\emph{From \href{../theories/html/hydras.MoreAck.PrimRecExamples.html}{MoreAck.PrimRecExamples}}.

\input{movies/snippets/PrimRecExamples/const0NIsPR}

\subsubsection{Proving that \texttt{plus} is primitive recursive}

The lemma \texttt{plusIsPR} is already proven in \href{../theories/html/hydras.Ackermann.primRec.html}{Ackermann.primRec}. We present in 
\href{../theories/html/hydras.MoreAck.PrimRecExamples.html}{MoreAck.PrimRecExamples}
a commented version of this proof, 

First, we look for lemmas which may help to prove that a given function obtained with the recursor \texttt{nat\_rec} is primitive recursive.

\input{movies/snippets/PrimRecExamples/PrimRecExamplesSearch}

The following lemma shows it suffices to prove that
Standard library's function \texttt{plus} is extensionally equal to a function defined with
\texttt{nat\_rec}.

\input{movies/snippets/PrimRecExamples/isPRExtEqualTrans}

Thus, let us define an helper and prove its equivalence with \texttt{plus}.

\input{movies/snippets/PrimRecExamples/plusAlt}


\vspace{4pt}


We are now able to complete the proof.

\input{movies/snippets/PrimRecExamples/plusIsPRa}


We already proved that \texttt{S} is \texttt{PR 1}, but we need to consider a function of three arguments, which ignores its first and third arguments.
Fortunately, the library \texttt{primRec} already contains lemmas adapted to this kind of situation.

\vspace{4pt}
\input{movies/snippets/PrimRecExamples/checkFilter0101IsPR}
\vspace{4pt}


Thus, our first subgoal is solved easily. The rest of the proof 
is just an application of already proven lemmas.

\vspace{4pt}

\input{movies/snippets/PrimRecExamples/plusIsPRb}


\begin{todo}
Comment more examples from   \href{../theories/html/hydras.MoreAck.PrimRecExamples.html}{MoreAck.PrimRecExamples}.
\end{todo}

\index{primrec}{Exercises}
\begin{exercise}
There is a lot of lemmas similar to \texttt{filter010IsPR} in the \texttt{primRec} library, useful to control the arity of functions.
Thus, the reader may look at them, and invent simple examples of application for each lemma.
\end{exercise}

\index{primrec}{Exercises}
\begin{exercise}
Multiplication of natural number is already proven in the \texttt{primRec} library. Write a proof of your own, then compare to the library's version.
\end{exercise}

%\input{movies/snippets/PrimRecExamples/justATest}

\subsubsection{More examples}

The following proof decomposes the \texttt{double} function as the composition of 
multiplication with the identity and the constant function which returns $2$.
\emph{Note that the lemma \texttt{const1\_NIsPR} considers this function as an unary function (unlike \texttt{const0\_NIsPR})}. 
\input{movies/snippets/PrimRecExamples/doubleIsPRa}



\index{primrec}{Exercises}
\begin{exercise}
Prove that the following functions are primitive recursive. 

\input{movies/snippets/MorePRExamples/factDef}

\input{movies/snippets/MorePRExamples/expDef}

\input{movies/snippets/MorePRExamples/tower2Def}


\textbf{Hint:} You may have to look again at the lemmas of the library
\href{../theories/html/hydras.Ackermann.primRec.html}{Ackermann.primRec} if you meet some difficulty.
You may start this exercise with the file
    \href{https://https://github.com/coq-community/hydra-battles/blob/master/exercises/primrec/MorePRExamples.v}{exercises/primrec/MorePRExamples.v}.
\end{exercise}



\index{primrec}{Exercises}
\begin{exercise}
Show that the function \texttt{min: naryFunc\,2} is primitive
recursive.

\emph{You may start this exercise with
    \href{https://https://github.com/coq-community/hydra-battles/blob/master/exercises/primrec/MinPR.v}{exercises/primrec/MinPR.v}.}

\end{exercise}

\index{primrec}{Exercises}

\begin{exercise}
Write a simple and readable proof that the Fibonacci function is primitive recursive.

\input{movies/snippets/FibonacciPR/fibDef}


\textbf{Hint:}  You may use as a helper the function which computes the pair \linebreak
$(\texttt{fib}(n+1),\texttt{fib}(n))$. 
Library \href{../theories/html/hydras.Ackermann.cPair.html}{Ackermann.cPair} contains
the definition of the encoding of $\mathbb{N}^2$ into $\mathbb{N}$, and the proofs that 
the associated constructor and projections are primitive recursive.
\emph{You may start this exercise with the file
    \href{https://github.com/coq-community/hydra-battles/blob/master/exercises/primrec/FibonacciPR.v}{exercises/primrec/FibonacciPR.v}.}

\end{exercise}

\index{primrec}{Exercises}
\begin{exercise}

Prove the following lemmas (which may help to solve the next  exercise).

\input{movies/snippets/isqrt/boundedSearch3}
\input{movies/snippets/isqrt/boundedSearch4}

\textbf{Note:}  The function \texttt{boundedSearch} is defined
in Library \href{../theories/html/hydras.Ackermann.primRec.html}{Ackermann.primRec}. 
\end{exercise}

\index{primrec}{Exercises}

\begin{exercise}
Prove that the function which returns the  integer square root of any natural number  is primitive recursive.

\emph{You may start this exercise with the file
    \href{https://github.com/coq-community/hydra-battles/blob/master/exercises/primrec/isqrt.v}{exercises/primrec/isqrt.v}.}

\end{exercise}

\section{Proving that a given function is \emph{not} primitive recursive}
\label{sect:ack-not-PR}

The best known example of a total recursive function which is not primitive recursive is the Ackermann function. We show how to adapt the classic proof (see for instance~\cite{planetmath}) to the constraints of \gallina. We hope this formal proof 
 is a nice opportunity to explore
the treatment of primitive recursive functions by R. O'Connor,
and to play with dependent types.

\subsection{Ackermann function}

Ackermann function is traditionally defined as a function from 
$\mathbb{N}\times \mathbb{N}$ into $\mathbb{N}$, through
three equations:

\begin{align}
A(0,n)&=n+1\\
A(m+1,0)&=A(m,1)\\
A(m+1,n+1)&=A(m,A(m+1,n))
\end{align}

Let us try to define this function in \coq{} (in curried form).

\input{movies/snippets/Ack/AckFixpointFail.tex}

A possible workaround is to make \texttt{m} be the 
decreasing argument, and define --- within \texttt{m}'s scope --- a local helper function which computes (\texttt{Ack m n}) for any \texttt{n}.
This way, both functions \texttt{Ack} and \texttt{Ackm} have a (structurally) strictly decreasing argument.

\input{movies/snippets/Ack/AckFixpointAlt.tex}

We prefered to define a variant which uses explicitely 
 the functional \texttt{iterate},
where (\texttt{iterate\,$f$\,$n$})
is the $n$-th iteration of $f$\,\footnote{Please not confuse with \texttt{primRec.iterate}, which is monomorphic and does not share the same order of arguments.}. It makes it possible to apply a few lemmas proved in 
\href{../theories/html/hydras.Prelude.Iterates.html}{Prelude.Iterates}, for instance about the monotony of the $n$-th iterate of a given function. 


\vspace{4pt}
\noindent
\emph{From \href{../theories/html/hydras.Prelude.Iterates.html}{Prelude.Iterates}}.
\index{hydras}{Library Prelude!iterate}

\input{movies/snippets/Iterates/iterateDef}

\input{movies/snippets/Iterates/iterateLeNSN}


Thus, our definition of the Ackermann function is as follows:

\vspace{4pt}
\noindent
\emph{From \href{../theories/html/hydras.MoreAck.Ack.html}{MoreAck.Ack}}.
\index{maths}{Ackermann function}
\index{primrec}{Ackermann function}

\input{movies/snippets/Ack/AckFixpointIterate.tex}


\index{hydras}{Exercises}

\begin{exercise}
The file \href{../theories/html/hydras.MoreAck.Ack.html}{MoreAck.Ack} presents two other definitions of the Ackermann functions based on the lexicographic ordering on $\mathbb{N}\times\mathbb{N}$.
Prove that the four functions are extensionnally equal.
\end{exercise}


\subsubsection{First properties of the Ackermann function}

The three first lemmas make us sure that our function 
\texttt{Ack} satifies the ``usual'' equations.

\input{movies/snippets/Ack/AckRewrite}


\vspace{4pt}

The order of growth of the Ackermann function w.r.t. its first argument is illustrated by the following equalities.

\input{movies/snippets/Ack/Ack1N}
\input{movies/snippets/Ack/Ack2N}
\input{movies/snippets/Ack/Ack3N}
\input{movies/snippets/Ack/Ack4N}


\begin{remark}
 The statements above can be rewritten in a more uniform way:

 \begin{quote}
   For $m\in 1..4$, $\texttt{Ack}\,m\,n = f_m\,(n+3)-3$, where 
   \begin{align*}
   f_1(n)=&\,n+2 \\
   f_2(n)=&\,n\times 2\\
   f_3(n)=&\,2^n\\
   f_4(n)=&\,2^{2^{\dots^2}}\quad(n\;\textit{levels})
   \end{align*}
 \end{quote}
\end{remark}


An important property of the Ackermann function helps us 
to overcome the difficulty raised by nested recursion, by climbing up the hierarchy $\texttt{Ack}\,n\,\_\;(n\in\mathbb{N})$.


\noindent
\emph{From \href{../theories/html/hydras.MoreAck.Ack.html}{MoreAck.Ack}}.

\input{movies/snippets/Ack/nestedAckBound}



Please note also that for any given $n$, the unary function
(\texttt{Ack\,$n$}) is primitive recursive.

\vspace{4pt}

\noindent

\emph{From \href{../theories/html/hydras.MoreAck.AckNotPR.html}{MoreAck.AckNotPR}}.

\input{movies/snippets/AckNotPR/AckNIsPR}





\subsection{A proof by induction on all primitive recursive functions}

Ìn order to prove that \texttt{Ack} (considered as a function of two arguments) is not primitive recursive, the usual method consists in two steps:


\begin{enumerate}
\item Prove that for any primitive recursive function $f:\mathbb{N}\rightarrow\mathbb{N}\rightarrow\mathbb{N}$, there exists some natural number $n$ depending on $f$, such that, for any $x$ and $y$, 
$f\,x\,y \leq \texttt{Ack}\,n\,(\textrm{max}\,x\,y)$ (we say that $f$ is \emph{``majorized''}  by \texttt{Ack}).
\item Show that \texttt{Ack} fails to satisfy this property.
\end{enumerate}

First, we prove that any primitive function of two arguments is majorized by \texttt{Ack}.
If we look at the inductive definition of primitive recursive functions, page~\pageref{def:Primrec}, it is obvious that a proof by induction on the construction of primitive recursive functions must consider functions of any arity.

The following scheme allows us to write proofs by induction on the class of primitive recursive functions. 

\vspace{4pt}
\noindent
\emph{From \href{../theories/html/hydras.Ackermann.primRec.html}{Ackermann.primRec}.}

\index{coq}{Commands!Scheme}

\input{movies/snippets/primRec/SchemePrimRecInd}



Please note that, in order to prove a property shared by any primitive recursive function of, say, arity 2, this induction scheme  leads you to consider an expension of the considered property to primitive recursive function of any arity.

Thus the lemma we will have to prove is the following one:


  \begin{quote}
    For any $n$, and any primitive recursive function $f$ of  arity $n$, there exists some natural number $q$ such that the following inequality holds:
 \[
  \forall x_1,\dots,x_n, 
      f(x_1,\dots,\,x_n)\leq\textrm{Ack}(q,\textrm{max}(x_1,\dots,x_n))
\]
 \end{quote}


But dots don't belong to \gallina's syntax! So, we may use \coq's vectors for denoting arbitrary tuples.

First, we extend \texttt{max} to vectors of natural numbers (using the notations of module \texttt{VectorNotations} and some more definitions from 
\href{../theories/html/hydras.Prelude.MoreVectors.html}{Prelude.MoreVectors}). So, (\texttt{t\,$A$\,$n$}) is the type of vectors of $n$ elements of type $A$, and the constants \texttt{cons}, \texttt{nil}, \texttt{map}, etc., refer to vectors and not to lists. Likewise, the notation \texttt{x::v} is an abbreviation for
\texttt{VectorDef.cons x \_ v}.

\index{coq}{Dependently typed functions}

\input{movies/snippets/MoreVectors/maxvDef}

\input{movies/snippets/MoreVectors/maxvLemmasa}

\input{movies/snippets/MoreVectors/maxvLemmasb}

\input{movies/snippets/MoreVectors/maxvLemmasc}


We have also to convert any application
$(f\,x_1\,x_2\,\dots\,x_n)$ into an application of a function 
to a single argument: the vector of all the $x_i$\,s.
This is already defined in 
Library~\href{../theories/html/hydras.Ackermann.primRec.html}{Ackermann.primRec}.


\input{movies/snippets/primRec/evalListDef}

Indeed, (\texttt{evalList $m$ $v$ $f$}) is the application to the vector $v$ of
an uncurried version of $f$.
In Library\href{../theories/html/hydras.MoreAck.AckNotPR.html}{MoreAck.AckNotPR}, we introduce a lighter notation.

\index{coq}{Dependently typed functions}

\input{movies/snippets/AckNotPR/vApply}



We are now able to translate in \gallina{} the notion of ``majorization'':

\index{coq}{Dependently typed functions}

\input{movies/snippets/AckNotPR/majorizedDefs}


Now, it remains to prove that any primitive function is majorized by \texttt{Ack}.
The three base cases  are as follows:

\input{movies/snippets/AckNotPR/majorSucc}

\input{movies/snippets/AckNotPR/majorZero}

\input{movies/snippets/AckNotPR/majorProjection}



The remaining cases are proved within a mutual  induction.

\index{coq}{Mutual induction}

\input{movies/snippets/AckNotPR/majorAnyPR}



\subsection{Looking for a contradiction}

The following lemma is just a specialization of \texttt{majorAnyPR} to
binary functions (forgetting vectors, coming back to usual notations).

\input{movies/snippets/AckNotPR/majorPR2}

We prove also a strict version of this lemma, thanks to the following property (proved in Library
\href{../theories/html/hydras.MoreAck.Ack.html}{MoreAck.Ack}~).

\input{movies/snippets/Ack/AckStrictMonoL}


\vspace{4pt}
\noindent
\emph{From \href{../theories/html/hydras.MoreAck.AckNotPR.html}{MoreAck.AckNotPR}.}


\input{movies/snippets/AckNotPR/majorPR2Strict}



If the Ackermann function were primitive recursive, then there would exist some natural number $n$, such that, for all $x$ and $y$, the inequality 
$\texttt{Ack}\,x\,y\leq \texttt{Ack}\,n\,(\texttt{max}\,x\,y)$ holds.
Thus, our impossibility proof is just a sequence of easy small steps.

\input{movies/snippets/AckNotPR/AckNotPR}

\begin{remark}
It is easy to prove that any unary function which dominatates \texttt{fun n => Ack n n} fails to be primitive recursive. To this end, we use an instance of \texttt{majorAnyPR} dealing with unary functions.

\vspace{4pt}
\noindent

\emph{From \href{../theories/html/hydras.MoreAck.AckNotPR.html}{MoreAck.AckNotPR}}.

\input{movies/snippets/AckNotPR/majorPR1}

Then, we write  a short proof by contradiction.

\input{movies/snippets/AckNotPR/domAckNotPR}

\end{remark}

\begin{remark}
Note that the Ackermann function is a counter-example to the (false) statement:
\begin{quote}
{\color{red}
  ``Let $f$ be a function of type \texttt{naryFunc\,2}. If, for any $n$, the fonction $f(n)$ is primitive recursive, then f is primitive recursive.''}
\end{quote}
\end{remark}


\section{The length of standard hydra battles}
\label{sect:battle-length-notPR}

The module \href{../theories/html/hydras.Hydra.Hydra_Theorems.html}{Hydra\_Theorems} contains a proof that the function which computes the length of standard hydra battles is not primitive recursive. More precisely, we consider, for a given hydra $h=\iota(\alpha)$, the length of a standard battle which starts with the replication factor $k$ (see Sect~\vref{def:L-alpha}).

This proof is  a little more complex than the preceding one.

\subsection{Definitions}

The function we consider is defined and proven correct in
Module~\href{../theories/html/hydras.Hydra.Battle_length.html}{Hydra.Battle\_length}.

\input{movies/snippets/Battle_length/BattleLength}

\subsection{Proof steps}

Now, let us assume that the function \texttt{l\_std} is primitive recursive.


\emph{From \href{../theories/html/hydras.Hydra.Hydra_Theorems.html}{Hydra.Hydra\_Theorems}}.

\input{movies/snippets/Hydra_Theorems/battleLengthNotPRa}

Let us consider the hydra represented by the ordinal $\omega^\omega$.

\input{movies/snippets/Hydra_Theorems/battleLengthNotPRb}


In order to get rid of the substraction in the definition of \texttt{l\_std}, we work with a helper function.

\input{movies/snippets/Hydra_Theorems/battleLengthNotPRc}

Under the hypothesis \texttt{H}, $m$ is also primitive recursive.

\input{movies/snippets/Hydra_Theorems/battleLengthNotPRd}


\subsubsection{Comparison between $F$ and $H'$}

In \href{../theories/html/hydras.Epsilon0.F_alpha.html}{Epsilon0.F\_alpha}, we prove a relation between the $F$ and $H'$ functionals. For any $\alpha$ and $k>0$,
$H'_{\omega^\alpha}(k)\geq F_\alpha(k)$.

\input{movies/snippets/F_alpha/HprimeF}


Our proof of this lemma is not trivial at all, it uses some properties of the Ketonen-Solovay's toolkit. We advise the reader to explore this proof, with the help of an ide or softwre like alectryon.
%%
%%% To move the path chapter
%%%%

% \begin{Coqsrc}
%   alpha : E0
%   IHalpha : forall beta : E0, beta o< alpha -> P beta
%   Halpha : Limitb alpha
%   n : nat
%   ============================
%   H'_ (Phi0 (CanonS alpha n)) (S n) <= 
%   H'_ (Phi0 (CanonS alpha (S n))) (S n)
% \end{Coqsrc}

% In mathematical notation: $H'_{\omega^{\canonseq{\alpha}{n}}}(n+1) \leq
% H'_{\omega^{\canonseq{\alpha}{n+1}}}(n+1)$.

% \vspace{4pt}

% But there exists no lemma saying that, if 
% $\beta\leq \alpha$, then $H'_\beta(k)\leq H'_\alpha(k)$, for any $\alpha$ and $\beta$. For instance, 
% $H'_{42}(3)=45> H'_\omega(3)=7$.


% Looking for lemmas of the form $H'_\beta(k)\leq H'_\alpha(k)$, we find this one (from our library
% \href{../theories/html/hydras.Epsilon0.Hprime.html}{Epsilon0.Hprime}):

% \begin{Coqanswer}
% H'_restricted_mono_l : 
%     forall (alpha beta : E0) (n : nat), 
%       Canon_plus n alpha beta -> 
%       H'_ beta n <= H'_ alpha n.
% \end{Coqanswer}

% Thus, it remains to prove that 
% there exists a path from ${\omega^{\canonseq{\alpha}{n+1}}}$
% to ${\omega^{\canonseq{\alpha}{n}}}$ composed of 
% $n+1$-steps.

% Fortunately, the Ketonen-Solovay machinery contains three lemmas which help us to build such a path.


% \begin{Coqanswer}
% KS_thm_2_4_lemma5 :
%   forall [i : nat] [alpha beta : T1],
%   const_pathS i alpha beta ->
%   nf alpha -> alpha <> zero -> 
%   const_pathS i (phi0 alpha) (phi0 beta)

% KS_thm_2_4 :
%   forall [lambda : T1], nf lambda ->limitb lambda ->
%   forall i j : nat, i < j -> 
%    const_pathS 0 (canonS lambda j) (canonS lambda i)

% Cor12_1 :
% forall [alpha : T1], nf alpha ->
%       forall (beta : T1) (i n : nat),
%       beta t1< alpha ->
%      i <= n -> const_pathS i alpha beta -> 
%      const_pathS n alpha beta
% \end{Coqanswer}
  
\subsubsection{End of the proof}

We finish the proof by comparing several fast growing functions.

\emph{From \href{../theories/html/hydras.Epsilon0.L_alpha.html}{Epsilon0.L\_alpha}}

\input{movies/snippets/L_alpha/HprimeL}

\emph{From \href{../theories/html/hydras.Epsilon0.F_omega.html}{Epsilon0.F\_omega}}


\input{movies/snippets/F_omega/FVsAck}



By transitivity, we get the inequality
$F_\omega(k+1)\leq m(k+1)$, for any $k$.

\input{movies/snippets/Hydra_Theorems/mGeFOmega}


We finish the proof by noting that the function $m$ (composed with \texttt{S}) dominates the Ackermann function, which leads to a contradiction.

\input{movies/snippets/Hydra_Theorems/mDominatesAck}

\input{movies/snippets/Hydra_Theorems/SmNotPR}

\input{movies/snippets/Hydra_Theorems/LNotPR}



