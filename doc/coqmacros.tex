\definecolor{termcolor}{rgb}{0.1,0.1,0.9}
\definecolor{prooftermcolor}{rgb}{0.3,0.1,1.0}
\definecolor{metavarcolor}{rgb}{0.5,0.0,1.0}
\definecolor{darkgreen}{rgb}{0.1,0.7,0.1}
\definecolor{answercolor}{rgb}{.8,.15,.08}
\definecolor{sourcecolor}{rgb}{.07,.1,.7}
\definecolor{normalcolor}{rgb}{0.0,0.0,0.0}
\definecolor{exbluecolor}{rgb}{0.1,0.1,0.9}
\definecolor{dontlookcolor}{rgb}{0.5,0.5,0.5}
\definecolor{termcolor}{rgb}{0.05,0.05,0.4}
\definecolor{lookcolor}{rgb}{0.9,0.1,0.0}
\definecolor{darklookcolor}{rgb}{0.5,0.1,0.0}
\definecolor{prooftermcolor}{rgb}{0.3,0.1,1.0}
\definecolor{typecolor}{rgb}{1.0,0.4,0.0}
\definecolor{taccolor}{rgb}{0.1,0.10,0.0}
\definecolor{pink}{rgb}{0.8,0.6,0.6}
\definecolor{darkmagenta}{rgb}{0.4,0.0,0.6}
\definecolor{darkblue}{rgb}{0.0,0.0,0.6}


%\newcommand{\todo}{\textbf{To do}}

\usepackage{xcolor,fancyvrb, mdframed}


%%% MMCG symbols

\newcommand{\idot}[1]{\mbox{$\bullet_{#1}$}}
\newcommand{\baredot}{\mbox{$\bullet$}}
\newcommand{\iback}[1]{\mbox{$\backslash_{#1}$}}
\newcommand{\back}{\mbox{$\backslash$}}
\newcommand{\islash}[1]{\mbox{$/_{#1}$}}
\newcommand{\icomma}[3]{\mbox{$(#1\,,\,#2)_{#3}$}}
%\newcommand{\icomma}[3]{\mbox{$#1\,\mathbin{\circ}_{#3}\,#2$}}
%\newcommand{\comma}{\mbox{$\mathbin{\circ}$}}
\newcommand{\comma}{\mbox{,}}

\newcommand{\slashe}{\mbox{\textbf{$/_{\textbf{E}}$}}}
\newcommand{\slashi}{\mbox{\textbf{$/_{\textbf{I}}$}}}
\newcommand{\backe}{\mbox{\textbf{$\backslash_{\textbf{E}}$}}}
\newcommand{\backi}{\mbox{\textbf{$\backslash_{\textbf{I}}$}}}

\newcommand{\dote}{\mbox{\textbf{$\baredot_{\textbf{E}}$}}}
\newcommand{\doti}{\mbox{\textbf{$\baredot_{\textbf{I}}$}}}
\newcommand{\diame}{\mbox{\textbf{$\Diamond_{\textbf{E}}$}}}
\newcommand{\diami}{\mbox{\textbf{$\Diamond_{\textbf{I}}$}}}
\newcommand{\struct}{\mbox{\textbf{struct}}}
\newcommand{\axrule}{\mbox{\textbf{Ax}}}
\newcommand{\boxe}{\mbox{\textbf{$\Box_{\textbf{E}}$}}}
\newcommand{\boxi}{\mbox{\textbf{$\Box_{\textbf{I}}$}}}
\newcommand{\AssD}{\textbf{L$_\Diamond$}}
\newcommand{\ComD}{\textbf{P$_\Diamond$}}

\newcommand{\marginok}[1]{\marginpar{\raggedright OK:#1}}
\newcommand{\tab}{{\null\hskip1cm}}
\newcommand{\Ltac}{\mbox{{$\cal L$}tac}}
\newcommand{\coq}{\mbox{{Coq}}}
\newcommand{\compcert}{\mbox{{CompCert}}}

\newcommand{\pcoq}{\mbox{{Pcoq}}}
\newcommand{\grail}{\mbox{{Grail}}}
\newcommand{\lcf}{\mbox{{LCF}}}
\newcommand{\hol}{\mbox{{HOL}}}
\newcommand{\pvs}{\mbox{{PVS}}}
\newcommand{\icharate}{\mbox{{Icharate}}}
\newcommand{\isabelle}{\mbox{{Isabelle}}}
%\newcommand{\coq}{\mbox{{Coq}}}
\newcommand{\prolog}{\mbox{{Prolog}}}
\newcommand{\goalbar}{\tt{}{\color{black}------------------------------------}\it}
\newcommand{\gallina}{\mbox{{Gallina}}}
\newcommand{\joker}{\texttt{\_}}
\newcommand{\eprime}{\(\e^{\prime}\)}
\newcommand{\Ztype}{\textbf{Z}}
\newcommand{\propsort}{\textbf{Prop}}
\newcommand{\setsort}{\textbf{Set}}
\newcommand{\typesort}{\textbf{Type}}
\newcommand{\ocaml}{\mbox{{OCaml}}}
\newcommand{\haskell}{\mbox{{Haskell}}}
\newcommand{\why}{\mbox{{Why}}}
\newcommand{\Pascal}{\mbox{{Pascal}}}

\newcommand{\ml}{\mbox{{ML}}}

\newcommand{\scheme}{\mbox{{Scheme}}}
\newcommand{\lisp}{\mbox{{Lisp}}}

\newcommand{\implarrow}{\mbox{$\Rightarrow$}}
\newcommand{\metavar}[1]{?#1}
\newcommand{\notincoq}[1]{#1}
\newcommand{\coqscope}[1]{\%#1}
\newcommand{\arrow}{\mbox{$\rightarrow$}}
\newcommand{\fleche}{\arrow}
\newcommand{\funarrow}{\mbox{$\Rightarrow$}}
\newcommand{\ltacarrow}{\funarrow}
 \newcommand{\coqand}{\mbox{\(\wedge\)}}
 \newcommand{\coqor}{\mbox{\(\vee\)}}
 \newcommand{\coqnot}{\mbox{\(\neg\)}}
% \newcommand{\coqand}{\texttt{/\textbackslash}}
% \newcommand{\coqor}{\texttt{\textbackslash/}}
% \newcommand{\coqnot}{\mbox{\(\neg\)}}
\newcommand{\hide}[1]{}
\newcommand{\hidedots}[1]{...}
\newcommand{\sig}[3]{\texttt{\{}#1\texttt{:}#2 \texttt{|} #3\texttt{\}}}
\renewcommand{\neg}{\mbox{$\sim$}}

%%% Operateurs, etc.
\newcommand{\impl}{\mbox{$\rightarrow$}}
\newcommand{\appli}[2]{\mbox{\tt{#1 #2}}}
\newcommand{\applis}[1]{\mbox{\texttt{#1}}}
\newcommand{\abst}[3]{\mbox{\tt{fun #1:#2 \funarrow #3}}}
\newcommand{\coqle}{\mbox{$\leq$}}
\newcommand{\coqge}{\mbox{$\geq$}}
\newcommand{\coqdiff}{\mbox{$\neq$}}
\newcommand{\coqiff}{\mbox{$\leftrightarrow$}}
\newcommand{\prodsym}{\mbox{\(\forall\,\)}}
\newcommand{\exsym}{\mbox{\(\exists\,\)}}
\newcommand{\included}{\mbox{\(\subseteq\)}}

\newcommand{\substsign}{/}
\newcommand{\subst}[3]{\mbox{#1\{#2\substsign{}#3\}}}
\newcommand{\anoabst}[2]{\mbox{\tt[#1]#2}}
\newcommand{\letin}[3]{\mbox{\tt let #1:=#2 in #3}}
\newcommand{\prodep}[3]{\mbox{\tt \(\forall\,\)#1:#2,$\,$#3}}
\newcommand{\prodplus}[2]{\mbox{\tt\(\forall\,\)$\,$#1,$\,$#2}}
\newcommand{\dom}[1]{\textrm{dom}(#1)} % domaine d'un contexte (log function)
\newcommand{\norm}[1]{\textrm{n}(#1)} % forme normale (log function)
\newcommand{\coqZ}[1]{\mbox{\tt{`#1`}}}
\newcommand{\coqnat}[1]{\mbox{\tt{#1}}}
\newcommand{\coqcart}[2]{\mbox{\tt{#1*#2}}}
\newcommand{\alphacong}{\mbox{$\,\cong_{\alpha}\,$}} % alpha-congruence
\newcommand{\betareduc}{\mbox{$\,\rightsquigarrow_{\!\beta}$}\,} % beta reduction
%\newcommand{\betastar}{\mbox{$\,\Rightarrow_{\!\beta}^{*}\,$}} % beta reduction
\newcommand{\deltareduc}{\mbox{$\,\rightsquigarrow_{\!\delta}$}\,} % delta reduction
\newcommand{\dbreduc}{\mbox{$\,\rightsquigarrow_{\!\delta\beta}$}\,} % delta,beta reduction
\newcommand{\ireduc}{\mbox{$\,\rightsquigarrow_{\!\iota}$}\,} % delta,beta reduction

\newcommand{\slam}[2] {\mbox{$\lambda_#1\;#2$}}
\newcommand{\srho}[2] {\mbox{$\rho_#1\;#2$}}
\newcommand{\Vforall}[2] {\mbox{$\forall\,#1\,\in\,V(#2)$}}
\newcommand{\Eforall}[2] {\mbox{$\forall\,#1\,\in\,E(#2)$}}
\newcommand{\Vforone}[2] {\mbox{$\exists!\,#1\,\in\,V(#2)$}}
\newcommand{\Eforone}[2] {\mbox{$\exists!\,#1\,\in\,E(#2)$}}
\newcommand{\Vforsome}[2] {\mbox{$\exists\,#1\,\in\,V(#2)$}}
\newcommand{\Eforsome}[2] {\mbox{$\exists\,#1\,\in\,E(#2)$}}


% jugement de typage
\newcommand{\these}{\mbox{$\boldsymbol{\large \vdash}$}}
\newcommand{\rthese}[1]{\mbox{$\boldsymbol{\vdash_{#1}}$}}
\newcommand{\replace}[2]{\mbox{$#1[#2]$}}
\newcommand{\msubst}[2]{\mbox{$#1\{#2\}$}}
\newcommand{\disj}{\mbox{$\backslash/$}}
\newcommand{\conj}{\mbox{$/\backslash$}}
\newcommand{\deriv}[2]{\mbox{$#1\;\these\;#2$}}
\newcommand{\smalljuge}[3]{\mbox{$#1 \these #2 \boldsymbol{:} #3 $}}
%\newcommand{\juge}[3]{\mbox{$#1 \these #2 \boldsymbol{:} #3 $}}
\newcommand{\goal}[3]{\mbox{$#1,#2 \these^{\!\!\!?} #3  $}}
\newcommand{\sgoal}[2]{\mbox{$#1\,\these^{\!\!\!\!?}\, #2 $}}
\newcommand{\sequent}[2]{\mbox{$#1\;\these\; #2 $}}

\newcommand{\reduc}[5]{\mbox{$#1,#2 \these #3 \rhd_{#4}#5 $}}
\newcommand{\convert}[5]{\mbox{$#1,#2 \these #3 =_{#4}#5 $}}
\newcommand{\convorder}[5]{\mbox{$#1,#2 \these #3\leq _{#4}#5 $}}
\newcommand{\wouff}[2]{\mbox{$\emph{WF}(#1)[#2]$}}




\newcommand{\type}{\boldsymbol{:}}

% jugement absolu

%\newcommand{\ajuge}[2]{\mbox{$ \boldsymbol{\vdash} #1 : #2 $}}
\newcommand{\ajuge}[2]{\mbox{$\these #1 \boldsymbol{:} #2 $}}

%%% logique minimale
\newcommand{\propzero}{\mbox{$P_0$}} % types de Fzero

%%% logique propositionnelle classique
\newcommand {\ff}{\boldsymbol{f}} % faux
\newcommand {\vv}{\boldsymbol{t}} % vrai

\newcommand{\verite}{\mbox{$\cal{B}$}} % {\ff,\vv}
\newcommand{\sequ}[2]{\mbox{$#1 \vdash #2 $}} % sequent
\newcommand{\strip}[1]{#1^o} % enlever les variables d'un contexte



%%% tactiques
\newcommand{\decomp}{\delta} % decomposition
\newcommand{\recomp}{\rho} % recomposition

%%% divers
\newcommand{\cqfd}{\mbox{\textbf{cqfd}}}
\newcommand{\fail}{\mbox{\textbf{F}}}
\newcommand{\succes}{\mbox{$\blacksquare$}}
%%% Environnements


%% Fzero
\newcommand{\con}{\mbox{$\cal C$}}
\newcommand{\var}{\mbox{$\cal V$}}

\newcommand{\atomzero}{\mbox{${\cal A}_0$}} % types de base de Fzero
\newcommand{\typezero}{\mbox{${\cal T}_0$}} % types de Fzero
\newcommand{\termzero}{\mbox{$\Lambda_0$}} % termes de Fzero 
\newcommand{\conzero}{\mbox{$\cal C_0$}} % contextes de Fzero 

\newcommand{\buts}{\mbox{$\cal B$}} % buts

%%% for drawing terms
% abstraction [x:t]e
\newcommand{\PicAbst}[3]{\begin{bundle}{\bf abst}\chunk{#1}\chunk{#2}\chunk{#3}%
                        \end{bundle}}

% the same in DeBruijn form
\newcommand{\PicDbj}[2]{\begin{bundle}{\bf abst}\chunk{#1}\chunk{#2}
                       \end{bundle}}


% applications
\newcommand{\PicAppl}[2]{\begin{bundle}{\bf appl}\chunk{#1}\chunk{#2}%
                         \end{bundle}}

% variables
\newcommand{\PicVar}[1]{\begin{bundle}{\bf var}\chunk{#1}
                          \end{bundle}}

% constantes
\newcommand{\PicCon}[1]{\begin{bundle}{\bf const}\chunk{#1}\end{bundle}}

% arrows
\newcommand{\PicImpl}[2]{\begin{bundle}{\impl}\chunk{#1}\chunk{#2}%
                         \end{bundle}}

%\newcommand{\todo}[1]{\textbf{Todo: #1}}

% %% texte Coq

% %% réponses de Coq
% \newenvironment{coqanswer}{\begin{mdframed}[topline=false, bottomline=false, leftline=false, rightline=false, skipabove=0pt, skipbelow=1pt, backgroundcolor=gray!20]\begin{alltt}\it}
%{\end{alltt}\end{mdframed}}




%%%% scripts coq
\newcommand{\prompt}{\mbox{\sl Coq $<\;$}}
\newcommand{\natquicksort}{\texttt{nat\_quicksort}}

%% ajoute des parenthèses légères aux expressions de la forme "f x y"

\newcommand{\coqparen}[1]{\mbox{\textcolor{gray}{(}\,\textbf{#1}\,\textcolor{gray}{)}}}

%%% texte Coq déjà parenthésé; pas la peine dans rajouter
\newcommand{\coqsimple}[1]{\mbox{\textbf{#1}}}
\newcommand{\safeit}{\it}
\newcommand{\byrule}[1]{\mbox{$\underset{\scriptstyle #1}{\longrightarrow}$}}
\newcommand{\byrules}[1]{\mbox{$\underset{\scriptstyle #1}{\longrightarrow^*}$}}

\newcommand{\backback}{\mbox{$\backslash\backslash$}}
\newcommand{\cons}{\mbox{\tt\,::\,}}
\newcommand{\nil}{\mbox{\tt\,[\,]\,}}



\usepackage{fontspec}

\usepackage{mathtools}
\usepackage{xcolor}
\usepackage{caption}
\usepackage{placeins}


\definecolor{mintedbgcolor}{rgb}{0.95,0.95,1.0}
\definecolor{badcoqcolor}{rgb}{0.8,0.8,0.8}
\definecolor{altcoqcolor}{rgb}{0.7,1.0,0.8}

\definecolor{mathcolor}{rgb}{0.95,0.90,0.85}

\definecolor{vertfluo}{rgb}{0.623, 0.94, 0.886}
\definecolor{lightred}{rgb}{1.0, 0.74, 0.7}
\definecolor{lightgray}{rgb}{0.7, 0.7, 0.7}



%\usepackage{fontspec}
\usepackage{tikz}
\usepackage{tikzsymbols}
\usetikzlibrary{arrows}
\usepackage{graphicx}
\usepackage{amsmath,mathdots,dsfont}
\usepackage{amssymb}

% handling vref in both latex and html
\newcommand{\myvref}[2]{\vref{#1}}
% unicode stuff
\usepackage{newunicodechar}
\newunicodechar{𝟙}{\ensuremath{\mathds{1}}}
\newunicodechar{ℤ}{\ensuremath{\mathds{Z}}}




\newcommand{\rounds}{\mbox{\,\texttt{-+->}\,}}
\newcommand{\round}{\mbox{\,\texttt{-1->}\,}}
\newcommand{\rplus}[1]{\mbox{$\,\underset{#1}{\longrightarrow}\,$}}
\definecolor{lookcolor}{rgb}{0.9,0.2,0.0}
\newcommand{\canonseq}[2]{\mbox{$\{#1\}(#2)$}}
\newcommand{\showmath}[1]{\mathcolor{\mbox{$#1$}}}
\newcommand{\bigmath}[1]{\textcolor{blue}{\mbox{\[#1\]}}}
\newcommand{\myemph}[1]{\textcolor{lookcolor}{\,\it{#1}\,}}
\usepackage{theorem}

{\theorembodyfont{\rmfamily}
 \newtheorem{exercise}{Exercise}[chapter]
 \newtheorem{project}{Project}[chapter]
\newtheorem{remark}{Remark}[chapter]
}
\newtheorem{theorem}{Theorem}[chapter]
\newtheorem{proposition}{Proposition}[chapter]

\newtheorem{lemma}{Lemma}[chapter]
\newtheorem{conjecture}{Conjecture}[chapter]
\newtheorem{definition}{Definition}[chapter]
\newtheorem{todo}{To do}[chapter]

\newmdenv[linecolor=red]{mathframe}

\newcommand{\mathcolor}[1]{\colorbox{mathcolor}{\mbox{#1}}}

%% ajoute des parenthèses légères aux expressions de la forme "f x y"

\newcommand{\kscite}[1]{\texttt{#1} {of} KS~\cite{KS81}}
\newcommand{\kpcite}[1]{\texttt{#1} {of} KP~\cite{KP82}}
\newcommand{\slcite}[1]{\texttt{#1} {of} TT~\cite{Sladek07thetermite}}

%% KS eaters
\newcommand{\gnaw}[2]{\mbox{$\{#1\}\langle #2 \rangle $}}
\definecolor{darkred}{rgb}{.3,.0,.0}
\definecolor{sourcecolor}{rgb}{.07,.1,.7}

%\input{setup-latex}

\DefineVerbatimEnvironment%
{Coqsrc}{Verbatim}
{fontsize=\small, frame=single,rulecolor=\color{blue},fillcolor=\color{blue!05}}


\DefineVerbatimEnvironment%
{Coqanswer}{Verbatim}
{fontsize=\small, frame=single,fontshape=it,rulecolor=\color{red},fillcolor=\color{red!05}}

\DefineVerbatimEnvironment%
{Coqbad}{Verbatim}
{fontsize=\small, frame=single,rulecolor=\color{black},fillcolor=\color{black!05}}


\DefineVerbatimEnvironment%
{Coqalt}{Verbatim}
{fontsize=\small, frame=single,rulecolor=\color{green},fillcolor=\color{green!05}}

\setcounter{secnumdepth}{5}
\DeclarePairedDelimiter{\floor}{\lfloor}{\rfloor}
\DeclarePairedDelimiter{\ceil}{\lceil}{\rceil}
