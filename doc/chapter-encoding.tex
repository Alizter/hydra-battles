\chapter{G\"{o}del's Encoding (in construction)}
\label{chap:encoding}


\section{Cantor pairing function}

The library \href{../theories/html/hydras.Ackermann.cPair.html}{Ackermann.cPair} defines and study Cantor's bijection from
$\mathbb{N}\times\mathbb{N}$ into $\mathbb{N}$.
Indeed the \texttt{cPair} function used in this library
 is slightly different 
     from the "usual" Cantor pairing function shown in  a big part 
     of the litterature , and Coq's standard library \footnote{In 
\url{https://coq.inria.fr/distrib/current/stdlib/Coq.Arith.Cantor.html}}.
       Since both versions are equivalent upto a swap of the 
      rguments [a] and [b], we still use  Russel O'Connors definitions and statements, mainly in order to not have to modify the order of sub-goals in long proofs.


      \subsection{A helper function}

      The following function computes the sum of all natural numbers between $1$ and $n$: $\Sigma_{i=1}^{i=n}\,i$.

      
      \inputsnippets{cPair/sumToNDef}

      \inputsnippets{cPair/sumToN1, cPair/sumToN1}

      The tools presented in Chapter~\ref{chapter:primrec} allow us to prove that \texttt{cPair} is primitive recursive.

      \inputsnippets{cPair/sumToNPR}
      
      \subsection{Cantor's pairing function}

      In the Ackermann/G\"{o}del projects, the Cantor pairing function is defined as below:

      $$\textrm{cPair}\,a\,b = a+ \Sigma_{i=1}^{i=a+b}\,i$$

      \inputsnippets{cPair/cPairDef}

      Figure~\ref{fig:cpair} shows a few values of
      \texttt{cPair\,$a$\,$b$ }, where $a$ is the line number and $b$ the column number.

      \begin{figure}[h]
        \centering

         \[
        \begin{array}[h]{c|cccccc}
          &0&1&2&3&4&\dots \\
          \hline 
          0&0&1&3& 6 & 10&\dots \\
          1&2&4&7&11& \dots& \\
          2&5&8 & 12& \dots&& \\
            3&9&13&\dots & && \\
          4&14&\dots&&&&\\
          \dots&\dots&&&&&\\
        \end{array}
        \]
        
        \caption{Cantor pairing function (first values)\label{fig:cpair}}
      \end{figure}


      \begin{remark}
        Compatibility with Standard lib's pairing function
        is stated
        by the equality \texttt{cPair $a$ $b$ = Cantor.to\_nat $(b,a)$} for any $a$ and $b$. Obviously, this swap and uncurrying of $a$ and $b$ doesn't change the fundamental properties of Cantor's pairing function (being a primitive recursive bijection, monotony properties).
      \end{remark}