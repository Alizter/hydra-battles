\chapter{G\"{o}del's Encoding (in construction)}
\label{chap:encoding}


\section{Cantor pairing function}

The library \href{../theories/html/hydras.Ackermann.cPair.html}{Ackermann.cPair} defines and study Cantor's bijection from
$\mathbb{N}\times\mathbb{N}$ into $\mathbb{N}$.
Indeed the \texttt{cPair} function used in this library
 is slightly different 
     from the "usual" Cantor pairing function shown in  a big part 
     of the litterature , and Coq's standard library \footnote{In 
\url{https://coq.inria.fr/distrib/current/stdlib/Coq.Arith.Cantor.html}}.
       Since both versions are equivalent upto a swap of the 
      rguments [a] and [b], we still use  Russel O'Connors definitions and statements, mainly in order to not have to modify the order of sub-goals in long proofs.


      \subsection{A helper function}

      The following function computes the sum of all natural numbers between $1$ and $n$: $\Sigma_{i=1}^{i=n}\,i$.

      
      \inputsnippets{cPair/sumToNDef}

      \inputsnippets{cPair/sumToN1, cPair/sumToN1}

      The tools presented in Chapter~\ref{chapter:primrec} allow us to prove that \texttt{cPair} is primitive recursive.

      \inputsnippets{cPair/sumToNPR}
      
      \subsection{Cantor's pairing function}

      In the Ackermann/G\"{o}del projects, the Cantor pairing function is defined as below:

      $$\textrm{cPair}\,a\,b = a+ \Sigma_{i=1}^{i=a+b}\,i$$

      \inputsnippets{cPair/cPairDef}

      Figure~\ref{fig:cpair} shows a few values of
      \texttt{cPair\,$a$\,$b$ }, where $a$ is the line number and $b$ the column number.

      \begin{figure}[h]
        \centering

         \[
        \begin{array}[h]{c|cccccc}
          &0&1&2&3&4&\dots \\
          \hline 
          0&0&1&3& 6 & 10&\dots \\
          1&2&4&7&11& \dots& \\
          2&5&8 & 12& \dots&& \\
            3&9&13&\dots & && \\
          4&14&\dots&&&&\\
          \dots&\dots&&&&&\\
        \end{array}
        \]
        
        \caption{Cantor pairing function (first values)\label{fig:cpair}}
      \end{figure}


      \begin{remark}
        Compatibility with Standard lib's pairing function
        is stated
        by the equality \texttt{cPair $a$ $b$ = Cantor.to\_nat $(b,a)$} for any $a$ and $b$. Obviously, this swap and uncurrying of $a$ and $b$ doesn't change the fundamental properties of Cantor's pairing function (being a primitive recursive bijection, monotony properties).
      \end{remark}

      \subsubsection{Main properties}

      In order to prove that the function \texttt{cPair} is primitive recursive, we express it as a composition of already proven primitive recursive functions.

      \inputsnippets{cPair/cPairIsPR}

      \texttt{cPair}'s injectivity is stated by two lemmas:

      \inputsnippets{cPair/cPairInj1, cPair/cPairInj2}

      \subsection{Projections}

      Let $a$ be some natural number, we look for two natural numbers $x$ and $y$ such that $\texttt{cPair}\,x\,y=a$. More we want to prove that the function which compute $x$ [resp. $y$] out of $a$ are primitive recursive.

      Let us show on a small example how these projections are defined. Let's take for instance $a=11$. Please look at the diagram of Fig.~\ref{fig:cpair}.

      The following function looks for the anti-diagonal $\{(x,y)|x+y=k\}$ the number $11$ belongs to, and returns $k$.

      \inputsnippets{cPair/searchXY}
      
      In our example, $11$ belongs to the anti-diagonal $\{(x,y)|x+y=4\}$, which contains values from $10$ to $14$.
      Thus the line containing $11$ is the line $x=11-10=1$, and the column is $y=4-1= 3$. Finally, we get  $11=\texttt{cPair}\,1\,3$.

       \inputsnippets{cPair/cPairPI12}

       The following lemmas (still from \href{../theories/html/hydras.Ackermann.cPair.html}{Ackermann.cPair}) prove the correctness of these projections.

       \inputsnippets{cPair/cPairProjections,
         cPair/cPairProjections1, cPair/cPairProjections2,
         cPair/cPairPi1IsPR, cPair/cPairPi2IsPR}
       
       Finally, let us show a few inequalities.

       \inputsnippets{cPair/cPairLe1, cPair/cPairLe1A,
         cPair/cPairLe2, cPair/cPairLe2A, cPair/cPairLe3,
         cPair/cPairLt1,cPair/cPairLt2}

       \subsection{List encoding}

       The encoding of a list of natural numbers is based
       on \texttt{cPair}, through a structural recursion.

         
       \inputsnippets{cPair/codeListDef}

      
       Let us look at the main step of the proof that \texttt{codeList} is injective.
       
  \inputsnippets{cPair/codeListInj}
  \inputsnippets{cPair/codeListInja}

  Applying \texttt{cPair}'s injectivity  and the induction hypothesis allows us to complete the proof.

  \subsubsection{Encoding the \texttt{nth} function}

  The following function allows us to compute the $n$-th element of a list of natural numbers, \emph{directly on the encoding of the considered list}.

  In the current version of the Ackermann Library, this function is defined \emph{via} an interactive proof. Its specification and
  correctness are proved by  separate lemmas. 
  

  \inputsnippets{cPair/codeNthDef}
  \inputsnippets{cPair/codeNthCorrect}
   \inputsnippets{cPair/codeNthIsPR}

   \index{hydras}{Exercises}
  \begin{exercise}
  Give a new equivalent definition of \texttt{codeNth}, without using tactics like \texttt{assert}.  You may define an help function for this purpose, in which case, please specify what your helper computes.
  \end{exercise}

  
     \index{hydras}{Exercises}
  \begin{exercise}
    Please consider the following definition:

    \inputsnippets{OnCodeList/membersDef}
    

   
    \noindent
    
    Prove that the functions \texttt{codeList} and \texttt{members} are inverse of each other.
  \end{exercise}

  \subsection{Strong recursion}

  List encoding helps us to define primitive recursive functions
  where the computation of $f\;n$ may  depend of
  [part of] the values $f\;0$, $f\;1$, \dots, $f\;(n-1)$.

  A simple example is the \texttt{div2} function
  defined by
  \begin{align*}
    \texttt{div2}\;0&=0\\
    \texttt{div2}\;1&=1\\
    \texttt{div2}\;(n+2)&=S(\texttt{div2}\;n)\\
  \end{align*}

  The trick is to define a helper $h$ where
  $h\;n\;a$ is the natural number which encodes
  the sequence $\langle \texttt{div2}(n-1),
  \texttt{div2}(n-2),\dots,\texttt{div2}\;1, \texttt{div2}\;0\rangle$.

  In \coq, the helper associated with \texttt{div2} is defined in
  \href{../theories/html/hydras.MoreAck/PrimRecExamples}{MoreAck.PrimRecExamples}.
  

  \inputsnippets{PrimRecExamples/fdiv2Def}
  
  
  The function \texttt{(evalStrongRec\;$n$\;$h$\;$c$)} computes
  $(f\;c)$ if $h$ is the helper associated with $f$.
  This function is defined in
  \href{../theories/html/hydras.Ackermann.cPair.html}{Ackermann.cPair}.

  \inputsnippets{PrimRecExamples/fdiv2Examples}

  Trying to compute the half of $5$ this way resulted in unbearably long computation times \dots.
  

  
  
   Definitions and related lemmas are quite tricky, but look easy to apply. 

   \inputsnippets{cPair/evalStrongRecDef}

   \inputsnippets{cPair/evalStrongRecPR}
    \inputsnippets{cPair/computeEvalStrongRecHelp}
  
  \index{ordinals}{Exercises}
    \begin{exercise}
      Prove formally that this implementation of \texttt{div2} is correct.
    \end{exercise}


   \index{ordinals}{Exercises}
    \begin{exercise}
     Define a function for computing the Fibonacci numbers by strong recursion.
    \end{exercise}

       