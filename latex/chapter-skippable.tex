\chapter{Introduction to ordinal numbers and the Coq proof assistant}

In this chapter, we introduce on a simple example the main concepts which are developped later. We assume the reader to have a very basic experience with the \coq{} proof assistant, for instance with one of the numerous tutorial one can consult on Internet
(see for instance \url{https://coq.inria.fr/documentation}). 

The definitions  and theorems we are going to show are just simpler versions (\emph{i.e.} restrictions) of the ones contained in the forthcoming chapters. We hope that this simplicity will help to understand how we represent, compute and reason about ordinal notations.

Beginners in \coq{} often play with natural numbers (the type \texttt{nat} of \coq's 
standard library), so we considered the ordinal  $\omega$ as ``well-known'' and chosed 
$\omega^2$ as a basis for our examples.



\section{Ordinal numbers}
\index{Maths!Ordinal numbers}

The proof of termination of all hydra battles presented in~\cite{KP82} is based
on \emph{ordinal numbers}.
From a mathematical point of view, an ordinal is a representant of an equivalence class for isomorphims of strict, total and well-founded orders.

We can also associate to every ordinal $\alpha$ a set whose elements are all ordinals strictly less than $\alpha$. Thus, it is meaningful  to consider \emph{finite}, \emph{infinite}, \emph{demunerable} and \emph{non-countable} ordinals.
The relation $<$ on ordinals is well-founded, and the order $\leq$ associated with
$<$ is total.

We cannot cite all the litterature published on ordinals since Cantor's book 
\cite{cantorbook}, and 
leave it to the reader to explore the bibliography. Let us cite the book by Schütte~\cite{schutte} which contains an axiomatic definition of the set of countable ordinals we used as a mathematical specification of our implementaion in \coq{}~\cite{CantorContrib}. 


Out of respect of the tradition, the meta-variables for ordinals will be 
 $\alpha$, $\beta$, $\gamma$, etc. 

\subsection{Definitions}

\begin{itemize}
\item Let  $\alpha$ be an ordinal; we say that  $\alpha$ is a \emph{successor} if there exists some ordinal  $\beta$ such that 
$\alpha$ is is the least ordinal strictly greater than  $\beta$.

\item We say that an ordinal $\lambda$ is a \emph{limit ordinal} is $\lambda$  is the least upper bound of a stricly increasing sequence of ordinals.
The meta-variable $\lambda$ will be used for denoting  limit ordinals.

\item \index{Maths!Transfinite induction}
An ordinal is either $0$, a limit ordinal or a successor ordinal. This case analysis, as well as \emph{transfinite} ({i.e.}, well-founded) induction is used in many proofs about ordinal numbers.

\item The segment of finite ordinals is isomorphic to the set of natural numbers.
 The first infinite ordinal is the limit ordinal $\omega$.
 
\item The operations $+$, $\times$ and exponentiation on $\mathbb{N}$ are extended on ordinals numbers. Note that these  extensions are not commutative any more.  For instance $\omega = 1 + \omega \not= \omega + 1$
and $\omega = 2 \times \omega \not= \omega \times 2$.

\item The ordinal $\epsilon_0$ is the least solution of the equation
 \(\alpha=\omega^{\alpha}\).
\end{itemize}


Please note that the set of countable ordinals is not countable.

 


\subsection{Ordinal notations}
In a proof assistant like \coq{},  it is useful to represent ordinals through some data-type, and make arithmetical operations and comparison effectively implemented  through certified functions.

Of course, since the set of all countable ordinals is not countable, it cannot be represented by a language of finite terms. 



\subsubsection{Notations systems for small ordinals}

The following list shows five popular notation systems for several ordinals (shown in increasing order).
Please keep in mind that a notation system for a given ordinal $\alpha$ represents 
ordinals \emph{strictly below} $\alpha$.

\begin{itemize}
\item $i\;(i\in\mathbb{N})$ \coq's type \verb@{j:nat | j < i}@ (see also
\href{https://www.math.nagoya-u.ac.jp/~garrigue/lecture/2018_AW/mathcomp-1.7.0/htmldoc/mathcomp.ssreflect.fintype.html#Ordinal}{finite ordinals in Mathcomp}).
\item $\omega$: \coq's type \texttt{nat}, with the order \texttt{Nat.lt}.
\item $\omega^n$ (for some  integer  $n\geq 2$) : the type of $n$-uples of natural numbers, with the lexicographic product of $n$ copies of \texttt{Nat.lt}.
\item  $\omega^\omega$: the set of nonincreasing sequences of natural numbers, lexicographically ordered (also the set of finite multisets of natural numbers).
\item $\epsilon_0$: The set of terms in Cantor normal form (see Chap.~\ref{chap/T1}).
\item $\Gamma_0$: The set of terms in Veblen normal form.
\end{itemize}


